%! TeX program = xelatex
\documentclass[a4paper,openany]{book}

% openany and oneside

\usepackage{pdfpages}
\usepackage[hidelinks]{hyperref}

\usepackage{xeCJK}
\usepackage{iitem}
\usepackage{fontspec}
\usepackage{xunicode}
\usepackage{xltxtra}
\usepackage{indentfirst}
\usepackage{geometry}
\usepackage{setspace}
\usepackage{array}
\usepackage[marginal]{footmisc}
\usepackage{graphicx}
\usepackage{makecell}
\usepackage[titletoc]{appendix}
\usepackage[shortlabels]{enumitem}
% \setlist[enumerate]{nosep}
% \setlist[enumerate]
% \setlist[itemize]

\renewcommand{\arraystretch}{1.5}

% \setmainfont{Stempel Garamond LT Pro}
\setmainfont{Bembo MT Pro}
\setCJKmainfont[BoldFont=FZDaBiaoSong-B06S,ItalicFont=FZNewKai-Z03S]{FZShuSong-Z01S}
% FZXiaoBiaoSong-B05S, FZDaBiaoSong-B06S
\setCJKsansfont{FZLanTingHeiS-R-GB}
\setCJKmonofont{FZShuSong-Z01S}
% \setCJKfamilyfont{content}{FZShuSong-Z01S}
% \newcommand{\shusong}{\setCJKfamilyfont{content}}
% \setCJKfamilyfont{title}{FZXiaoBiaoSong-B05S}
% \newcommand{\biaosong}{\setCJKfamilyfont{title}}
% \setCJKfamilyfont{strong1}{FZKai-Z03S}
% \newcommand{\kaiti}{\setCJKfamilyfont{strong1}}

% \setenumerate[1]{itemsep=0pt,partopsep=0pt,parsep=\parskip,topsep=5pt}
% \setitemize[1]{itemsep=0pt,partopsep=0pt,parsep=\parskip,topsep=5pt}
% \setdescription{itemsep=0pt,partopsep=0pt,parsep=\parskip,topsep=5pt}

\setenumerate{itemsep=0pt,partopsep=0.25pt,parsep=0em,topsep=0pt}
\setitemize{itemsep=0pt,partopsep=0.25pt,parsep=0em,topsep=0pt}
\setdescription{itemsep=0pt,partopsep=0pt,parsep=0em,topsep=0pt}

\setlength{\parindent}{2em}
\setlength{\parskip}{0.75em}
\renewcommand{\baselinestretch}{1.25}

\usepackage{fancyhdr}
\usepackage{lastpage}
\pagestyle{fancy}
\renewcommand{\headrulewidth}{0.4pt}
\renewcommand{\footrulewidth}{0pt}
\fancypagestyle{plain}{
	\pagestyle{fancy}
}

\fancyhead[RE]{\textit{\leftmark}}
\fancyhead[LO]{\textit{江南大学团务中心}}
\fancyhead[LE,RO]{\textit{第} \  \thepage\  \textit{页}}
% \cfoot{\textit{第\ \thepage\ 页}}
% \cfoot{——\ \thepage\ ——}
\cfoot{}

% \renewcommand{\contentsname}{目录} 
% \renewcommand{\appendixname}{附录}

 \geometry{left=3.174cm,right=3.174cm,top=2.54cm,bottom=2.54cm}

\title{\huge \textit{江南大学第十届模拟联合国大会} \\ \Huge \textbf{学术手册}}
\author{江南大学团务中心  \\  \small Powered By \XeLaTeX}

\begin{document}

% \includepdf{1.pdf} 插入封面

\begin{titlepage} % Suppresses displaying the page number on the title page and the subsequent page counts as page 1
	
	\raggedleft % Right align the title page
	
	\rule{1pt}{\textheight} % Vertical line
	\hspace{0.05\textwidth} % Whitespace between the vertical line and title page text
	\parbox[b]{0.75\textwidth}{ % Paragraph box for holding the title page text, adjust the width to move the title page left or right on the page
		
		{\large\bfseries 江南大学第十届模拟联合国大会 \\[1\baselineskip] \Huge 学术手册}\\[2\baselineskip] % Title
		{\large\textit{Powered By \XeLaTeX}}\\[4\baselineskip] % Subtitle or further description
		{\large{第十届江南大学模拟联合国大会组委会}} % Author name, lower case for consistent small caps
		
		\vspace{0.5\textheight} % Whitespace between the title block and the publisher
		
		{\noindent \textit{江南大学团务中心}}\\[\baselineskip] 
	}

\end{titlepage}

\thispagestyle{empty}

\frontmatter
\thispagestyle{empty}
\tableofcontents

\mainmatter

\chapter{模联概况}

模拟联合国 (Model United Nations),简称模联(MUN),是世界各国官方和民间团体特意为青年人组织的活动。青年学生们扮演各个国家的外交官,以联合国会议的形式,通过阐述观点、政策辩论、投票表决、做出决议等亲身经历,熟悉联合国的运作方式,了解世界发生的大事对他们未来的影响,了解自身在未来可以发挥的作用;同时也是对联合国大会和其它多边机构的仿真学术模拟,是为青年人组织的公民教育活动。在活动中,青年学生们扮演不同国家或其它政治实体的外交代表,参与围绕国际上的热点问题召开的会议。代表们遵循议事规则,在会议主席团的主持下,通过演讲阐述观点,为了“国家利益”辩论、磋商、游说。他们与友好的国家沟通协作,解决冲突;通过写作决议草案和投票表决来推进国际问题的解决。在模拟联合国,青年学生们通过亲身经历熟悉联合国等多边议事机构的运作方式、基础国际关系与外交知识,并了解世界发生的大事对他们未来的影响,了解自身在未来可以发挥的作用。

模联是一项历史悠久、开展广泛的学生活动。早在国际联盟时代,就有美国学生举办模拟国联会议的活动。1945年联合国的成立引起了世界人民的广泛热情。1951年,美国加州大学率先成立模拟联合国团队,举办模拟联合国活动。模拟联合国活动从此开始流行起来,不久便被推广到全世界,备受欢迎。许多大学甚至中学都建立了模拟联合国社团,举办和参加模拟联合国大会。

经过70多年的发展,模拟联合国活动现在已经风靡全世界,形式多样,规模不一,有国际大会、全国大会,还有地区级和校际间的大会,参与者有大学生到高中生,乃至初中生。同时,模拟联合国活动已经不仅仅是对联合国机构的模拟,它还包括对其他全球或地区性多边组织,政府内阁,国际论坛等组织或者会议的模拟。目前全世界每年有近四百个国际模拟联合国大会在五大洲的50多个国家召开。每年参与大会的师生来自世界100多个国家,总人数超过四百万人。

\section{联合国与模拟联合国}

联合国全球传播部迈出了改进模联的第一步,于2009年、2010年以及2011年分别在瑞士日内瓦、马来西亚吉隆坡和韩国仁川举办了全球模拟联合国大会,提供最佳实践的示范。通过成功举办这三次活动,联合国认识到要改进支持模联的方式,决定把重点放在为组织模联会议的全球学生领袖和指导老师举办工作坊和培训项目上。

自2012年起,联合国全球传播部开办了一系列工作坊,旨在帮助学生和指导老师了解以下事项:大会的议事规则;大会委员会会议的讨论和决议过程;如何起草和审议决议;大会和联合国秘书处官员的义务和责任,以及如何在模联中把握这两大主要机构领导结构上的关系;还有制定决策过程中达成一致的重要性和实现方法。全球传播部举办此类工作坊主要是为了以新方法培训模联项目负责人。这些方法更准确地反映了联合国运作的方式,加深学生对外交技能的认知,以及外交技能在联合国工作中发挥的重要作用。

\section{国内发展历程}

\subsection{总体发展}

对中国的模拟联合国活动追根溯源,最早的开拓者应当是北京顺义国际学校。这所学校专门服务于在华外国人子女。早在1993年,北京顺义国际学校就召开了第一届北京模拟联合国大会(Beijing Model United Nations,简称BEIMUN),该大会是海牙国际模拟联合国大会的分会。但是它主要的参与者是各国以及中国国内的国际学校,独立于国内教育界,因此也没有任何中国籍的学生参加过。真正意义上的中国高校首次模拟联合国会议诞生在1995年的外交学院,首次高中生模拟联合国会议是2005年的北京大学全国中学生模拟联合国会议。

模拟联合国在中国的发展可以分为四个阶段。第一个阶段是2001年至2004年,第二个阶段是2005年至2006年,第三个阶段是2007年至2010年,第四个阶段是2010年至今。

2001年至2004年,模拟联合国只在大学生中小范围开展,未能形成规模,也缺乏经验和指导,发展水平较低。而到了2005年,经过积淀,已经有大规模的高水平会议出现,同时模拟联合国走入高中生。在2007年至2008年,受到奥运的影响,中国对国际化愈发推崇,模拟联合国的知名度也迅速提高。在这个阶段,参与者的能力热情普遍较高,大量的精英从模拟联合国走向世界。

但到了2008年以后,模拟联合国逐渐泛化。除了一些老牌的优秀校际会议,大量的新生代高校和高中也开始举办自有会议,这使得模拟联合国的知名度急剧上升,但平均水平却同时下降。在这个阶段,一些会议开始进行本地化探索,推出了中文会议。在2008年后,以中文或者中英双语为工作语言的会议大量出现。模拟联合国,特别是高中生模拟联合国的本地化,降低了参与的门槛,使得模拟联合国走向大众。但同时,也使得这一原冠名着“精英”的活动受到来自大众的挑战。

\subsection{大学发展}

\subsubsection{2001年至2004年——模拟联合国的起步期}

模拟联合国在西方国家发展得比较成熟,到了90年代,这项活动才进入中国。20世纪以来发展势头尤为迅猛。北京大学、外交学院和西北工业大学于2001年先后成立了北京大学模拟联合国协会,成为全国高校中首批专门开展模拟联合国活动的学生组织。

2001年是模拟联合国正式在中国扎根的关键时期。2月份北京大学国际关系学院的两名学生远赴美国波士顿,观摩哈佛全美模拟联合国大会。两位同学回来之后被这一活动的独特吸引力所震撼,随即开始组建团队筹办北大校内的模拟联合国活动。5月份召开了北京大学首届模拟联合国会议暨《气候变化框架公约》缔约国大会,随后北京大学模拟联合国协会成立。同年,中国联合国协会推荐了4名外交学院的学生参与俄罗斯联合国协会在莫斯科举办的模拟联合国大会,该校在已实质开展活动六年后成立了模拟联合国协会。12月西北工业大学的模拟联合国团队成立。这三所学校成为最早在中国开展模拟联合国活动的高校。

从2002年开始,三所高校开始陆续派出团队参加国际的模拟联合国会议。当年5月中国联合国协会和外交学院共同主办了“2002北京模拟联合国大会”,共有首都15所高校81名代表参加,这成为国内第一个校际间的模拟联合国会议。随后在2004年,外交学院又承办了“亚太地区模拟联合国-千年发展目标”和“模拟人权委员会”,前者共有包括中国、美国、澳大利亚、俄罗斯、印度、香港和澳门等20多个地区的31支高校代表队参与了本次活动,时任联合国秘书长\textit{科菲·安南}\footnote{科菲·安南(Kofi Atta Annan,1938年4月8日-2018年8月18日),加纳库马西人,联合国第七任秘书长。科菲·安南在1997年1月1日年至2006年12月31日两个任期内,巩固了联合国在国际事务中的地位,促进了多边主义的进一步发展。他倡导集体安全、全球团结、人权法治,维护联合国的价值观念和道德权威。}也发来贺信。

从2001年到2004年,模拟联合国在中国处于起步摸索的阶段,学生们主要是参加会议积累做代表的经验,尝试学习国际会议,在国内开展同类活动。仅有若干所学校随之建立起模拟联合国的学生组织,大部分学校依然把此项活动作为学生会、研会等学生组织偶尔参与的非主流活动。

\subsubsection{2005年至2006年——模拟联合国的巩固期}

该阶段全国有近五十所高校都开展了模拟联合活动,已有模拟联合国团队的学校进入了巩固期,在全国处于一个平稳的发展时期。有的高校将模拟联合国活动纳入某一学生社团或学生会的活动;而后起的人民大学、北京师范大学、北京外国语大学、南京大学、东北师范大学等高校则也建立了专门的模拟联合国学生组织。香港、台湾地区高校的也在积极开展此项活动,香港浸会大学、台湾大学,台湾政治大学成立了相应的学生社团。该阶段国内校际间的模拟联合国以200人以下的中小型为主。

\subsubsection{2007年至今——模拟联合国的蓬勃发展时期}

中国大学生模拟联合国大会主要有三大会议,一是由中国联合国协会(China United Nations Association)主办的中国模拟联合国大会,二是由外交学院主办的\textit{北京模拟联合国大会}\footnote{自2017年起,原“北京模拟联合国大会(BMUN)”正式更名为“北京国际模拟联合国大会(BIMUN)”}(Beijing Model United Nations),三是由北京大学模拟联合国协会主办的亚洲国际模拟联合国大会(Asian International Model United Nations, Peking University)。另外,由厦门大学主办的海峡两岸模拟联合国大会(CrossStrait Model United Nations Conference),由浙江大学主办的浙江大学泛长三角地区模拟联合国大会(Zhejiang University PanYangtze River Delta Model United Nations Conference, PYDMUN),由清华大学和圆梦青年国际文化交流(北京)有限责任公司联合举办的圆梦·清华大学模拟联合国大会(OMUN with Tsinghua University Model United Nations, OMUN.THU)也是国内重要的模联大会。同时,由南京财经大学举办的江苏省首届模拟联合国大会(JSMUN2012)也吸引了来自全国各地众高校的优秀模联人。此外,由南京航空航天大学举办的江苏省大学生模拟联合国大会(Jiangsu Undergraduate Model United Nations)也逐步发展起来,成为较有影响力的模联大会。2012年,由华侨大学主办的第四届海峡西岸模拟联合国大会( WestStraits Model United Nations Conference, WSMUN)在厦门举行,也具有一定影响力。

在这一时期,模拟联合国不仅在东部发达地区进一步发展,更是深入了西部、西南部地区,使更多的学生有机会参与到这一国际活动中。

\chapter{江南大学模拟联合国大会概况}

模拟联合国(Model United Nations),简称模联(MUN),是对联合国大会和其它多边机构的仿真学术模拟,是为青年人组织的公民教育活动。在活动中,青年学生们扮演不同国家或其它政治实体的外交代表,参与围绕国际上的热点问题召开的会议。代表们遵循议事规则,在会议主席团的主持下,通过演讲阐述观点,为了“国家利益”辩论、磋商、游说。他们与友好的国家沟通协作,解决冲突;通过写作决议草案和投票表决来推进国际问题的解决。在模拟联合国,青年学生们通过亲身经历熟悉联合国等多边议事机构的运作方式、基础国际关系与外交知识,并了解世界发生的大事对他们未来的影响,了解自身在未来可以发挥的作用。

江南大学自2011年12月10日举办首届模拟联合国大会至今,已成功举办了九届模拟联合国大会。每届模拟联合国大会均由共青团江南大学委员会主办,由江南大学团务中心承办,并邀请校团委老师、各学院学生会主席、学生组织骨干届时列席观摩。

江南大学模拟联合国大会是非竞赛性质的学生活动,鼓励参与者之间的交流、沟通与合作。在模拟联合国会议中虽然会进行“最佳代表”等奖项的评选,但其目的主要在于鼓励热爱模拟联合国活动并在会议中表现出色的参会者,奖项并不具有实质性的功利意义。参加模拟联合国活动的经历要比奖项更加重要,模拟联合国活动的目的在于使参会者得到全面的发展以及综合能力、素质的提高,同时使参与者具有世界公民的意识以及增强对不同文化的理解与包容。

我们不仅在校内举办模联大会,还积极对外交流。在2015年,多名优秀的江大模联代表参加了全国各大高校举办的模联大会,并在其中斩获多个奖项。

\section{江南大学第一届模拟联合国大会}

江南大学第一届模拟联合国大会于2011年12月10日在北区大学生活动中心F210拉开序幕,来自各个学院的31个代表团围绕“北极石油开采问题”开展讨论。会议中,代表们从自身国家利益出发,针对石油开采问题提出本国的诉求,通过辩论与适度的妥协形成同盟,展现了良好的外交风度与谈判能力。代表们通过无数次动议,与各个国家结成联盟,通过不断地摩擦碰撞与协商,最终起草4份决议草案并进行了对决议草案的表决。

本次模拟联合国大会共评出最佳代表奖、最具魅力发言人奖、最具魅力大使奖、最佳立场文件奖4项奖项。

\section{江南大学第二届模拟联合国大会}

2012年11月25日,以“Enjoy MUN, Enjoy Professional”为主题的江南大学第二届模拟联合国大会开幕。本次会议的议题为“应对气候变化——国际责任与义务”。此次会议邀请了各学院学生会主席、学生组织骨干列席观摩,校团委副书记刘长青等三位老师出席指导会议。此次模联会议实行双代表制,会议语言为中英双语,大会共有50余个代表国家出席,包括美国、加拿大、俄罗斯、日本、英国、法国等。此次大会与会代表均为江南大学本科生或研究生。会议中,代表们从自身国家利益出发,针对环境问题提出本国的诉求,通过辩论与适度的妥协形成同盟,展现了良好的外交风度与谈判能力。经过了六小时的激烈辩论和磋商,大会圆满结束。

本次模拟联合国大会共评出最佳代表奖、最具魅力发言人奖、最具魅力大使奖、最佳立场文件奖4项奖项。

\section{江南大学第三届模拟联合国大会}

2013年11月16日上午,江南大学第三届模拟联合国大会开幕。本次模拟联合国大会由共青团江南大学委员会主办,由江南大学团务中心·MUN组委会承办。大会邀请了校团委书记刘长青,副书记王晖、魏珍吉以及各学生组织骨干出席。

本次大会出席代表由本科生、研究生、留学生组成,共计78人,代表39个国家,包括美国、英国、朝鲜、巴西、伊拉克等联合国第三委员会成员。本次会议实行双代表制,会议语言为中英双语。

校团委书记刘长青致词并宣布本次大会开幕。经过代表投票表决,大会议题定为《战争中的平民保护》。本次大会产生三个单项奖和两个团队奖:化工学院张笑晗同学荣获最杰出代表奖、研究生院桑苇同学获最具魅力大使奖、机械学院苗硕同学获优秀志愿者奖;外语学院沈雅如同学和Agyemang同学(留学生)获最佳代表团奖、纺服学院孟秋迪同学和物联网工程学院徐乃昊同学获最佳立场奖。

\section{江南大学第四届模拟联合国大会}

2014年5月17日上午8:30,江南大学第四届模拟联合国大会在北活音乐厅正式召开,会前收到来自郑州大学等高校的贺信及祝贺视频。本次会议的议题为“国际援助的有效监管”。此次会议邀请了各学院学生会主席、学生组织骨干列席观摩,实行双代表制,会议语言为中英双语。出席本次会议的国家共计33个,包括美国、日本、韩国、埃及等,50余位国家代表在会场上为维护本国利益展开激烈辩论,展现了良好的外交风度与谈判能力。经过了六小时的激烈辩论和磋商,大会圆满结束。体现出模拟联合国大会交流互动和专业的精神。

\section{江南大学第五届模拟联合国大会}

2015年4月24日上午8:30,江南大学第五届模拟联合国大会在北活F210正式召开,会前收到来自郑州大学等高校的贺信及祝贺视频。本次会议的议题为“航空安全与空难救援”。此次会议邀请了各学院学生会主席、学生组织骨干列席观摩,实行双代表制,会议语言为汉语。来自18个学院的70余为代表组成35个国家维护本国利益展开激烈辩论,在责任的号召下披荆斩棘,展现了良好的外交风度与谈判能力。

经过了一天的激烈辩论和磋商,大会圆满结束,并评选出最佳代表、最佳风采、最佳立场文件、最佳志愿者等奖项。

\section{江南大学第六届模拟联合国大会}

2016年5月7日上午8:30,江南大学第六届模拟联合国大会在北活F210正式召开,议题为“难民的救助与安置”。本次会议加入了全新的会议机制,引入了MPC会场(新闻媒体中心)。大会实行双代表制,会议语言为汉语。来自众多学院的62位代表组成了31个国家代表团,怀着悲悯与同情,为更好地解决人类共同面临的难民问题而会集,寻求最好的解决方案。此外,还有5位来自不同媒体的记者,从观察者的角度,准备做好历史的记录者。经过了一天的激烈辩论和磋商,大会圆满结束,代表们充分发扬了外交官的精神与风度,媒体人也尽自己的最大努力为人们展示会议的原貌。

本次会议共评选出最佳代表、最佳代表团、最佳文件写作、最受欢迎男/女代表等奖项。

\section{江南大学第七届模拟联合国大会}

2017年5月6日上午8:30,江南大学第七届模拟联合国大会在北区大学生活动中心F210正式召开,议题为“《武器贸易条约》的修订”。大会实行双代表制,会议语言为汉语。在开幕式上,本届大会组委会特别邀请到了江南大学校团委副书记魏珍吉老师为大会致开幕辞。在会场上,来自12个学院的44位代表组成22个国家代表团,对现有《武器贸易条约》中存在的问题展开了激烈的讨论与磋商。他们在会场上充分展现了外交官的国际风采,最终在投票阶段前提交了两份决议草案。虽然两份决议草案都未获得通过,但代表们在会场中展现出的外交精神与风度,仍值得所有人的称赞。

在闭幕式上,本次会议组委评选出最佳代表、最佳代表团、最佳文件写作以及最佳角色扮演等奖项。在颁奖典礼上,由江南大学校团委书记王维老师,江南大学校团委副书记魏珍吉老师以及江南大学团务中心指导老师王宏斌老师出席为获奖代表颁奖。

\section{江南大学第八届模拟联合国大会}

2018年5月12日,江南大学第八届模拟联合国大会开幕。本次会议议题为“战争中的人道主义问题”。大会实行双代表制,工作语言为汉语。在本次大会中,来自14个学院的58名代表组成了29个国家代表团对当前世界的因各种冲突而产生的人道主义问题进行了激烈的讨论;而组成3个媒体代表团的9名代表,作为整场会议的记录者,忠实地记录着会场上的发生的一切。在整天的激烈磋商与交涉中,代表们展现出了充分的外交精神,媒体人们也尽职尽责地以客观的视角记录了全天的会议。

在闭幕式上,本次大会学术团队评选出最佳代表团奖、最佳会议推动奖、最佳文件写作奖、最佳记者奖、最佳主编奖及荣誉提名奖。

\section{江南大学第九届模拟联合国大会}

2019年5月11日,江南大学第九届模拟联合国大会在江南大学文浩馆正式开幕。本次会议启用中英文双会场制度。中文场的议题为“新能源的推广和发展”,英文场的议题为“打击毒品犯罪”。本次大会一共设置了联合国环境规划署、国际刑警组织与主媒体中心三个委员会,大会采用双代表制,工作语言为中文和英文。会议正式阶段,来自各个学院的本科生和研究生共99人,代表41个国家就新能源推广和国际毒品犯罪等问题展开了深刻的讨论与磋商,充分展现了外交官的风采。大会设置的五个媒体机构的代表们在会议期间担任了优秀的旁观者和记录者,在社论中直截了当的指出会场仍存在的一些问题,具有极高的新闻素养。

经过了两天的紧张辩论,会议中达成了四份文件和多份国家之间的谅解备忘录。在最后的闭幕式上,外联总监王一淞先生和团务中心主任杨莞青女士分别对本次会议作出了总结。各委员会主席团对各自会场进行了总结并评选出优秀国家代表奖、优秀代表团奖、优秀记者奖等奖项。

\chapter{模拟联合国常规议事规则}

模拟联合国议事规则是模拟联合国会议召开所遵循的程序与规则。

一般情况下,会议包括一般性辩论与投票两个阶段。同时,危机(危机状态)也被视为会议中的一个特殊阶段,但并非每一个会议都有。

\section{点名与确定议题(会前阶段)}

\subsection{点名(Roll Call)}

在每小节会议开始之前,会议助理应按国名首字母顺序依次点出国家名(中文委员会一般以汉语拼音首字母或国名笔画数排序),被点到的国家应举起国家牌,并回答“出席”。如果代表未及时出席,需到场后用\textit{意向条}\footnote{为了维持会场秩序,开会期间代表如有任何问题或者需要进行游说、沟通,都要通过书写并传递意向条的方式向其他代表或主席团成员表达。意向条的书写需要符合该委员会的工作语言要求。}告知主席团。被点国家应为出席国,即包含理事国与观察国。点名后,会议助理应计算简单多数、三分之二多数、百分之二十和应出席国家数、出席国家数及缺席国家数通报全场。

\textbf{特殊说明}: 点名属主席团职权,主席可在任何其认为必要的会议阶段作点名。

\subsection{确定议题(Setting Topic)}

在多议题会议中(一般情况为双议题),代表们通过讨论、投票,确定出要首先讨论的议题。

\section{一般性辩论阶段}

\subsection{正式发言名单(Speakers’ List)}

确定议题之后,一般性辩论开始。主席会请有意愿发言的国家举牌并随机点取,发言的顺序即主席点名的顺序。当代表听到自己国家被点到之后,便放下国家牌。每个国家默认有120秒的发言时间(但可被更改,在“动议”中说明),待全体有意愿发言的国家被点名后,将产生正式发言名单。如需要追加发言(国家未在正式发言名单上或已经完成发言),代表可向主席团传意向条要求在发言名单上添加其代表的国家,主席会将该国添加在发言名单的最后。如果代表已在发言名单上,并且还没有发言,则不能在其发言之前追加发言。一旦发言名单上所有国家已发言,并且无任何国家追加发言,会议直接进入投票表决阶段。

\textbf{特殊说明}:当一国已举牌但主席在结束点名后未点到,或已点到但该国却未出现在正式发言名单时,该代表应及时举牌提问,否则视作放弃。

\subsection{让渡(Yield)}

让渡是指代表正式辩论阶段发言结束后所用时间仍有剩余且超过15秒时,代表把剩余时间用作其它事情的行为。让渡方式如表\ref{Yield}。

\begin{table}[ht]
\setlength{\belowcaptionskip}{5pt}
\caption{让渡方式及说明}
\label{Yield}
\centering
\begin{tabular}{| c | l |}
\hline
让渡形式 & \makecell[c]{说明} \\
\hline
让渡给主席   &  \multicolumn{1}{m{10cm}|}{一旦发言国将剩余时间让渡给主席,即意味着代表自动放弃剩余时间,主席将继续主持会议。}             \\
\hline
让渡给评论   &  \multicolumn{1}{m{10cm}|}{当代表将剩余时间让渡给评论,主席会请需要评论的代表举牌,并随机点出代表进行评论,让发言代表没有权利再一次进行观点的陈述或对评论进行反驳。这种让渡方式有一定风险,因为即使已经与盟国进行沟通,也不能保证主席所点出的进行评论的代表观点与发言代表完全一致。}   \\
\hline
让渡给他国代表 & \multicolumn{1}{m{10cm}|}{即“让渡给指定国家”,由发言国指定让渡国家,主席将请该国以剩余时间作台上发言。}             \\
\hline
让渡给问题   & \multicolumn{1}{m{10cm}|}{一旦发言国将剩余时间让渡给问题,主席会请需要提问的代表举牌,并随机点出代表进行提问,提问时间不占用剩余时间,提问内容必须针对发言的内容,发言国在剩余时间内回答。} \\
\hline
\end{tabular}
\end{table}

\textbf{特殊说明}:让渡仅存在于正式发言。剩余时间少于或等于15秒时,发言国不需选择让渡方式,其剩余时间默认让渡给主席。

\subsection{动议(Motion)}

动议,指由代表提出的改变现有会议进程的建议,需要全体代表投票表决。这些动议的内容可以是进行磋商、暂停会议(休会)、介绍决议草案或者进入投票阶段。动议种类及说明见表\ref{Motion}。

当一位代表按照发言名单的顺序发言完毕后,主席会询问场下有无问题和动议,此时代表可根据自身需要举牌提出动议。

\begin{table}[ht]
\setlength{\belowcaptionskip}{5pt}
\caption{动议种类及说明}
\label{Motion}
\centering
\begin{tabular}{| c | l | c |}
\hline
动议名称     & \makecell[c]{动议内容}       &  通过所需票数    \\
\hline
动议有主持核心磋商  & \multicolumn{1}{m{6cm}|}{代表通过动议有主持核心磋商来进行议题的深入讨论。有主持核心磋商讨论的主题、总时间、各代表发言时间由提出此动议的代表提出。总时长应可整除发言席位。 }                                                         & 简单多数      \\
\hline
动议自由磋商     & \multicolumn{1}{m{6cm}|}{提出自由磋商动议的代表需要规定时间。一旦动议获得通过,在规定时间内代表可以离开座位,更为密切地和盟友们交换意见。}   & 简单多数      \\
\hline
动议更改发言时间 & \multicolumn{1}{m{6cm}|}{代表可以通过动议更改发言时间,来重置每位代表在发言名单中的发言时间。}     & 简单多数      \\
\hline
动议暂停会议(或休会)      &   \multicolumn{1}{m{6cm}|}{ 代表动议暂停会议(或休会)可以结束会议的一个阶段,并使会议进入休息时间。}   & 简单多数      \\
\hline
关闭正式发言名单 & \multicolumn{1}{m{6cm}|}{初次表决时主席团应询问有否赞成与反对各一次,全体国家皆需投票,若此次全票通过则该动议通过;反之应再次询问,并从两边各随机点取一国,各有30秒阐述己方观点,此后再次投票,以三分之二多数通过之。} & 全票或三分之二多数\\
\hline
动议停止辩论 & \multicolumn{1}{m{6cm}|}{动议结束辩论与动议暂停会议有所不同。当结束辩论的动议生效后,会议将进入投票阶段。此动议一般在代表认为立场已经得到充分阐述,且决议草案已较为完善的情况下提出。}  &  简单多数 \\
\hline
\end{tabular}
\end{table}

\textbf{特殊说明}:发言与发言之间如主席团无特殊规定或说明则默认有三次无论通过与否的动议机会,主席团也可规定在会议开始的前几个正式发言间先不接纳动议,但需事先说明。此外,在危机状态下,非安理会者可有特殊动议(于第\pageref{Crisis}页的\ref{Crisis}小节中说明)。

\subsection{问题(Point)}

任何与会人员在没有人作台上发言(主席团的发言不属于台上发言,除非主席团成员因特殊原因前往发言台发言)时,可向主席团以口头形式提出问题;以书面(即意向条)形式提出问题时,不受任何限制。问题的种类及说明如表\ref{Point}。

\begin{table}[ht]
\setlength{\belowcaptionskip}{5pt}
\caption{问题种类及说明}
\label{Point}
\centering
\begin{tabular}{| c | l  |}
\hline
问题种类 & \makecell[c]{说明}  \\
\hline
程序性问题 & \multicolumn{1}{m{10cm}|}{关于一个委员会运行方式的问题。即对于提出方不明晰或提出方认为错漏的程序进行提问。} \\
\hline
咨询性问题 & \multicolumn{1}{m{10cm}|}{当代表对于会议程序有不明白的地方时,可以在台上没有代表发言时举牌向主席咨询。} \\
\hline
个人特权问题 & \multicolumn{1}{m{10cm}|}{当代表在会场上感觉有任何身体上的不适时,可以提出个人特权问题。} \\
\hline
\end{tabular}
\end{table}

\textbf{特殊说明}:问题仅可向主席团提出,同时,主席团不能无视问题的提出。

\subsection{附议(Second)}

附议,即赞同一个被提出的动议。此处特指在作程序性表决之前的一项程序,由主席团接纳某动议后作询问,认为该动议有讨论价值(无论本国是否支持/反对该动议所指的内容)即可举牌附议,动议国不得自行附议。在获得一国或以上的附议后,主席团才将该动议提交表决,否则该动议不需表决,直接否决。

\subsection{程序性表决(Procedural Vote)}

程序性表决即任何非针对文件的表决,一般以主席团询问有否赞成(特殊动议如“关闭正式发言名单”等还需询问有否反对),代表举牌投票,以该表决对象的法定票数通过。

\section{投票(Vote)}

投票是代表们表示是否支持一个针对委员会提出的行动的时刻。有两种类型:程序性和实体性。

正式发言名单已关闭或名单上的发言国已尽后,一般性辩论结束,进入投票阶段,此时享有投票权的应仅为理事国,表决对象应为决议草案,表决完毕后由主席团进行统计并通报。表决采取唱名表决,有两种投票方式,如表\ref{Vote}。

\begin{table}[ht]
\centering
\setlength{\belowcaptionskip}{5pt}
\caption{投票种类及说明}
\label{Vote}
\begin{tabular}{| c | l |}
\hline
投票方式 & \makecell[c]{说明}  \\
\hline
单式投票(Single Pattern) & \multicolumn{1}{m{8cm}|}{各理事国仅可选择投赞成、反对或弃权三者中任一者一次。} \\
\hline
叠式投票(Double Pattern) & \multicolumn{1}{m{8cm}|}{各理事国可于首轮投票中选择赞成、反对、弃权或过四者中任一者一次,次轮投票仅针对首轮投票中选择过的理事国,其仅可选择赞成或反对两者中任一者一次。} \\
\hline
\end{tabular}
\end{table}

\textbf{特殊说明}:投票前如主席团无规定或说明则默认为单式投票。

\section{危机(Crisis)}
\label{Crisis}

危机是会议进行当中,各国代表需要立即处理的具有不同危急程度的突发性事件。模联会议中的危机一般由主席团在会前设置好,事件可能与代表们讨论的议题相关。事件的形式有若干种,可能是突发事件的新闻报道,可能是国际组织的文件,可能是相关人员的视频资料,也可能是外交官派出国政府发来的外交指令等。

代表应通过各类方式(如代表团内部讨论、撰写指令性文件、会议平台集体行动)对危机进行商议和处理。代表无需对获得的所有危机做出应对,但需自行承担相关风险。

\subsection{程序性危机(Procedural Crisis)}

程序性危机分为一般及严重两种,一般情况下为严重程序性危机(Serious Procedural Crisis)的代称,对两种程序性危机的说明如表\ref{Procedural Crisis}。

\begin{table}[ht]
\setlength{\belowcaptionskip}{5pt}
\caption{程序性危机种类及说明}
\label{Procedural Crisis}
\centering
\begin{tabular}{| l | l | l |}
\hline
\makecell[c]{危机种类}      & \makecell[c]{内容}         & \makecell[c]{处理方式}           \\
\hline
\multicolumn{1}{|m{2cm}|}{一般程序性危机(Ordinary Procedural Crisis)}  & \multicolumn{1}{m{2cm}|}{应出席国出现缺席情况}          & \makecell[c]{不需处理}           \\
\hline
\multicolumn{1}{|m{2cm}|}{严重程序性危机(Serious Procedural Crisis)}  & \multicolumn{1}{m{2cm}}{触发“百分之二十原则”(Twenty- Percent Principle)}  & \multicolumn{1}{|m{8cm}|}{宣布进入“无限期休会”( Indefinitely Adjourn)后,全体出席国相当于进入无限期的自由磋商,即可下位讨论但不得随意离开会场。当缺席国向主席团通报出席并经计算后得缺席国未触发“百分之二十原则”,则主席当即召集全体在场代表回座并重新点名。若统计后未触发“百分之二十原则”,则会议继续进行;若依旧触发则再次进入“无限期休会”。} \\
\hline
\end{tabular}
\end{table}

\textbf{特殊说明}:“百分之二十原则”也称“隐形原则”(Invisible Principle),即任一执行本原则的会议在出席国及理事国数量同时未达五分之四(80\%)以上的前提下,即进入无限期休会。也就是说,譬如某次会议有30个国家,其中15个理事国、15个观察国,若此时有6个国家离席,在缺席的出席国数量上达到20\%,但若此6国中理事国数量未达3个或以上,则缺席理事国数量未达20\%,“百分之二十原则”依然不触发(同样,即使15个观察国全部缺席,但会议依然可进行,但若有6个理事国缺席,则触发),“百分之二十原则”的触发需满足出席国和理事国两个前提条件。但需注意的是,某些会议没有设置观察国,即全体出席国即理事国,此时相当于满足一个条件,则两个条件都满足了,从而“百分之二十原则”被触发。同时,当主席宣布落座重新点名后,若发现原来有出席的一些国家因各种原因又缺席了,只要满足了条件,依旧触发“百分之二十原则”。

\subsection{事态性危机(Situational Crisis)}

事态性危机即常称的“危机”,以事态发展之形式进行者,具体由会议主办方规定,在此不作说明。一般情况下,由主席裁决(Adjudicate)是否进入危机状态,若裁决进入则会议停止讨论原议题,转而讨论危机。同时,正式发言名单暂时中断,只能以动议形式作会议进程,此时可选择的动议方式包括有主持核心磋商、自由磋商及非安理会的委员会代表可选择动议“回到正式发言名单”(Back to the Speakers’ List),但这一项需以三分之二多数通过(仅表决一次),通过后即不讨论危机,回到一般性辩论阶段。

此外,非安理会的委员会主席团也可设置“危机结点”(Crisis Node),即宣布当指定时刻来临时,有指令草案则马上表决,无则自动回到一般性辩论。当然,代表也可选择在指定时刻之前解决危机。但需注意的是,当危机出现更新时,主席团可修改原设置的“危机结点”,否则不得改动。

\chapter{模拟联合国会议文件}

主席团有权在会前要求与会代表在指定时间内提交建议案等会议文件。

\section{立场文件(Position Paper)}

立场文件是模拟联合国会议讨论的基础文件,它反映了“各国”针对会议所讨论问题的原则立场,并对如何解决上述问题提出本国的意见。代表需要在会前立场文件书写完成并提交至主席团,供代表互相了解立场,更有针对性地准备会议,并在此基础上形成决议草案。它也是一国代表正式发言的基调或参考材料。立场文件要力图真实、完整地反映模拟国家的立场,它的提出需要队员们协同一致对模拟国家内外政策进行大量的调研工作。

\subsection{立场文件的内容}

立场文件不是八股文,其内容的安排顺序和文章结构都可以自己决定。另外,应尽量使文体显得官方和正式,这要求代表在写作的时候于选词、句式方面进行特别注意。在与主题相关的必要时候提供数据,举出实例,而不是空讲政策和态度,并用脚注或尾注来标明所引用的资料,在具体立场及措辞方面都要符合本国的实际情况。

\begin{enumerate}

\item 本国在该议题中的基本立场与态度;

\item 本国赞成及签署的与该议题相关的国际协议,本国参与的与该议题相关的国际合作和行动;

\item 本国与该议题的相关程度,对于解决该问题在国内曾经采取过的重要行动、通过的重要法案;

\item 本国对于解决该问题所提出的相关建议(包括国内措施和国际行动);

\item 本国领导人及政要发表的有关该议题的重要讲话;

\item 本国在该议题中的相关利益总结及立场底线总结。

\end{enumerate}

\newpage

\subsection{立场文件范例}

\vspace{1em}

\begin{spacing}{1}
\setlength{\parskip}{0em}

\noindent 代表:**\\
学校:****\\
国家:坦桑尼亚共和国\\
委员会:经济与社会理事会\\
议题:克隆人中的伦理道德
 
\vspace{1em} 
 
自从1997年克隆羊“多利”诞生以来,有关克隆人的合法性及合理性的讨论就没有停止过。克隆人技术在医学领域的广泛需求与其对道德伦理的冲击已构成一对日渐尖锐的矛盾。不解决克隆人的法律与道德问题,必将影响克隆技术的正常发展,甚至会导致一场严重的伦理危机的爆发。坦桑尼亚认为,一项能够确保克隆技术安全、稳定地向前发展国际性的法律的制定工作已势在必行。

坦桑尼亚作为联合国中的“最不发达国家”,目前还没有能力从事克隆技术的研究。但坦桑尼亚在克隆人技术对道德伦理的侵犯上已经与各国达成广泛共识,即:坦桑尼亚政府坚决反对克隆人,同时不赞成、不支持、不允许、也不接受任何克隆人试验。坦桑尼亚愿与各国一道为尽早实现对“禁止生殖性克隆人”的立法做出努力。

坦桑尼亚认为,相对于生殖性克隆人违反人类繁衍的自然法则,损害人类作为自然的人的尊严,引起严重的道德、伦理、社会和法律问题的影响,治疗性克隆与其有着本质的不同,它并不产生严重的道德、伦理、社会或法律问题,如在严格管理和控制下,也不能损害人类尊严。相反,治疗性克隆对于挽救人类生命,增进人类身体健康却有广阔前景和深厚潜力,如把握得当,可以造福人类。因而,我们反对将两个性质不同的问题混为一谈,也反对盲目禁止对克隆技术的研究。坦桑尼亚认为,问题的关键在于建立一定的规范,利用该项技术为人类造福。鉴于此,坦桑尼亚完全支持并将严格执行第59届联合国大会上达成的关于“禁止生殖性克隆人研究”的决议,即:联合国成员应禁止任何生殖性克隆行为和相关技术的研究活动,对任何试图通过克隆技术手段达到人类生命繁殖目的的克隆行为,应视为违法并严令取缔。

坦桑尼亚认为:只有建立完善的对治疗性克隆的法律规定及相应的监督机制,从而确保体细胞提供者的知情权,确保胚胎细胞究只停留在不具备生命意义的阶段(即未长出神经元),确保不将克隆的胚胎细胞植入人体子宫内,进而确保克隆技术在不违背伦理道德、不侵犯人类尊严的前提下造福人类,如此,才有助于实现全球范围内的治疗性克隆合法化。坦桑尼亚愿与各国一道,为促进治疗性克隆这一将极大地造福人类的技术的发展而不懈努力。

\end{spacing}

\newpage

\section{工作文件(Working Paper)}

工作文件是大会讨论成熟或就一个议题的某一部分讨论成熟时,即每位代表已阐述完基本观点后,由一个集团提交给主席的文件。在文件中应概述该集团对此问题统一立场、希望及解决方案。仅要求代表们在各自立场文件的基础上综合他国立场和要求,草拟出针对某一问题的看法和解决办法,不要求特定格式及投票。一般每次会议都会形成若干个集团,集团中的国家可以是同地域的、同民族的或是追求同种利益的。简而言之,工作文件就是一份由部分国家代表或某一利益集团提出的关于如何解决问题的想法的文件。工作文件通常是一份决议草案的雏形。它的内容包括针对所讨论议题的基本立场以及解决问题的建议与措施。工作文件没有固定形式,它可以分条列出,也可以用图表等形式表达 。

工作文件的主要用途是帮助起草国理清思路,明确立场。除了起草国,其它国家可以通过一集团的工作文件来了解本阶段讨论的进程及成果,明确会议形势以及该集团与己方的共同意见和分歧,这样便于双方进一步的的磋商和合作。工作文件有利于推进大会进程,使大会进入一个明确的讨论范围,避免分散,为决议草案的写作作准备。

工作文件一般是在会议进行到一半或更久时提交,届时主席团将宣布接收工作文件。经主席团审阅后合格的工作文件会由会议工作人员散发给各位代表。

\subsection{工作文件范例}

\vspace{1em}

\begin{spacing}{1}
\setlength{\parskip}{0em}

\centerline{\textbf{\large 工作文件1.1}}

\vspace{1em}

\noindent 委员会:联合国气候变化框架公约成员方会议 \\
议题:面对关于《京都议定书》的分歧,如何努力抑制全球气候变暖。 \\
起草国:中国

\vspace{1em}

\begin{enumerate}
\setlength{\itemsep}{0pt}
\setlength{\parsep}{0pt}
\setlength{\parskip}{0pt}

\item \textit{要求}有关国家以全人类的福祉为依归,承担起应负的责任,放弃与全世界对抗的立场,改变目前短视错误的做法,为全人类计,为子孙后代计,切实努力,克服困难,在促成京都议定书的通过方面发挥积极的作用。 

\item \textit{要求}国际社会在促使某些国家改变消极态度方面坚定立场,坚持努力,发挥整体的作用。 

\item \textit{建议}考虑对某些不负责任的国家采取一致的行动,以改变目前议定书暗淡的前景,以减少并最终杜绝此类事件的再次发生。这些行动的范围应该较为广泛,并应在此过程中强调联合国的作用。 

\item \textit{决定}所有议定书缔约方均应: 

\iitem \textit{建议}无论是发展中国家还是发达国家,都应承诺在现在的基础上加大对环境保护的投入,包括发展环保产业、增加对环境保护的研究经费、对民众进行环境教育等方面。 

\iitem \textit{加强}环境保护技术的研究、转让和共享。尤其是有利于温室气体排放减少和降解的技术,对这方面的技术保护应当给以不同于一般技术的特殊对待。 

\item \textit{提议}由各国派代表设立专门的机构和专家顾问委员会,以专门讨论实施联合国气候变化框架公约及其京都议定书的具体措施。 

\item \textit{鼓励}发展中国家积极采取措施减慢气候排放增加量,并在发展本国工业的同时尽量考虑到环境保护的因素。 

\end{enumerate}
\end{spacing}
\newpage

\section{决议草案(Draft Resolution)}

决议草案是由代表草拟、为委员会所讨论的议题寻求解决方案的一种文件格式。如果被投票通过,那么决议草案就成为正式决议。在一个委员会中,有关同一个议题的决议草案只能通过一份。

决议草案是按照联合国决议文件形式起草的对该议题的解决办法,是会议中提出的工作文件的加工和完善。一份决议草案可以由一个国家起草,也可以由多国起草。某决议草案的起草国不能再成为另一决议草案的起草国或附议国。一份决议草案需要得到与会代表国的20\%的签署才可以提交大会审议通过。

\subsection{决议草案格式}

\begin{enumerate}

\item 草案标题。

包括委员会、议题、决议草案编号(所在委员会的主席团给出)以及起草国和附议国的名单。

\item 草案正文。

草案正文是一个长句,中间的若干内容都是用逗号或者分号隔开,在草案结束时才出现句号。草案正文包括\textbf{序言性条款}和\textbf{行动性条款}两部分。

\iitem 序言性条款

序言性条款为决议中描述之前与议题有关的行动以及决议的必要性的部分。一般以\textbf{分词或者形容词}开头,并用斜体标注。

在这个部分中,主要回顾该议题的历史以及过去已经形成的关于此议题的决议和条约,讨论该议题的必要性。此外还可以包括联合国宪章、联合国秘书长或其他联合国机构领导人在此问题上的发言等内容。

\iitem 行动性条款

行动性条款为决议中描述模拟联合国大会将如何解决问题的部分,以\textbf{动词第三人称单数}(比如决定,成立,建议等等)开头。

行动性条款是真正涉及到会议实质部分的内容,在这一部分中,列举本次会议的讨论成果、措施以及建议。每一款内容都以动词开头,以分号结尾。这些条款都按照一定的逻辑顺序排列,用阿拉伯数字标注其顺序。每一款内容之包括一个建议或措施,每一款内容还可以包括逐条说明,用英文小写字母或者罗马数字标注。

\end{enumerate}

\subsection{唱名表决}

对决议草案的表决为唱名表决。主席团首先点名确认到场的国家代表,随后被点到国家名的代表需表决赞成、反对或弃权。对于一个议题,一个委员会只能通过一个决议草案,一旦有一份决议草案获得通过,将不会再对未表决的决议草案进行投票。

\subsection{决议草案的写作}

会议表达的最常见形式是决议。决议有特定的格式。每个决议由一个长单句组成。决议开篇是通过决议的主要机关的名字(例如,大会或安全理事会)。

\subsubsection{序言性条款}

之后是若干序言性条款。这些不是真正的段落,而是句子中的各个从句。每个序言性条款都以动词或动词性短语(例如“\textit{回忆到}、\textit{考虑到}、\textit{注意到}”)开头,以逗号结尾。有时,从句以多个关键词开头,例如:\textit{满意地注意到}、\textit{遗憾地注意到}等。这些词总是\textit{斜体}\footnote{请读者注意,本文中使用\textit{楷体}替代斜体。}。

序言性条款有助于解释执行段落要求采取的行动的基础。序言性条款可以用来搭建一个论点,也可以用于支撑论点。有时,序言性条款体现了一般原则,语气可以抬高。

\begin{table}[ht]
\setlength{\belowcaptionskip}{5pt}
\caption{序言性条款开头常用的措辞}
\centering
\begin{tabular}{| p{11cm} |}
\hline
\textit{承认、确认、感谢、批准、清楚、牢记、相信、建议、关切、了解、意识到、考虑到、确信、希望、强调、期待、表示、十分清楚、根据、已经通过、已经考虑、已经注意到、已经审查、谨慎地注意到、赞成地注意到、关切地注意到、满意地注意到、观察到、意识到、回顾、承认、寻求、考虑、强调、欢迎、然而}等。\\
\hline
\end{tabular}
\end{table}

关于序言性条款章节段落的排序,若序言提到《联合国宪章》,应置于前面。若决议开头笼统提及“《联合国宪章》的宗旨和原则”,则序言中应另起一条,指出《联合国宪章》阐释原则与决议主题相关性的具体章节或条款。第一次在序言段部分或者执行部分提到时应称之为《联合国宪章》,之后可以简写为《宪章》。

然后,与以往的决议或决定相关的内容放在第二位。对于安全理事会的决议,第一次提到需要提及日期,在此之后只需提及决议编号和年份,例如:第338(1973)号决议。

另外,可以包括对决议的内容或目的的一般性意见,这是其余文本的基础。这有助于为决议执行部分的行动呼吁奠定基础。

\subsubsection{行动性条款}

序言性条款之后是行动性条款,每个行动性条款都以动词或动词性短语开头,除最后一段以句号结束外,其他均以分号结束。与序言性条款相似,行动性条款中的关键词也需用\textit{斜体}标示。

\begin{table}[ht]
\setlength{\belowcaptionskip}{5pt}
\caption{行动性条款开头常用的措辞}
\centering
\begin{tabular}{| p{11cm} |}
\hline
\textit{接受、通过、同意、申请、批准、授权、呼吁、建议、考虑、决定、声明、决心、指示、强调、鼓励、支持、表示感激、表示希望、邀请、注意、赞许地注意到、关切地注意到、满意地注意到、宣布、重申、建议、提醒、重申、要求、决心、建议、支持、注意、敦促}等。\\
\hline
\end{tabular}
\end{table}


所有决议至少有一个行动性条款,但通常有多个序言性条款和多个行动性条款。如果多个序言性条款或行动性条款以相同的措辞开头(例如“\textit{注意到}”或“\textit{注意}”),通常第二次用“\textit{进一步}”,第三次及之后用“\textit{也}”来表示此类用途(例如,“\textit{注意到}”,“\textit{进一步注意到}”,“\textit{也注意到}”等)。一般情况下序言性条款没有编号,而行动性条款有。但是,如果只有一个行动性条款,则没有编号。

行动性条款表明会议决定采取的行动。精准的语言会加强政治影响力,促进执行。同样,也提倡简洁,因为它对政治影响力更大。

\subsubsection{检查清单}

在你提交一项决议草案之前,确保:

\begin{enumerate}
\item 你的代表团一致同意提交决议草案;
\item 你的决议得到其他代表团的支持。这不仅包括你通常联合的人,还包括支持决议推力的其他人并已经了解成功的几率;
\item 你已经和共同提出决议的代表团全程进行协商,他们对最终文本感到满意,并愿意成为共同提案国;
\item 你对该决议的文字充满信心,对内容和表达都比较满意;
\item 你已提前向模联秘书处或委员会主席团提交决议草案,以便他们能够在提出决议之前将文本分发给所有代表团。
\end{enumerate}

\newpage

\subsection{决议草案范例}
\begin{spacing}{1}
\setlength{\parskip}{0em}

\centerline{\textbf{\large 决议草案:1.1}}

\vspace{1em}

\noindent 委员会:联合国大会第一委员会——裁军与国际安全委员会 \\
\noindent 议题:无核国家的核威胁 \\
\noindent 起草国:阿根廷、中国、法国、德国、印度、美国 \\
\noindent 附议国:巴西、匈牙利、日本

\vspace{1em}

\noindent 大会,\\
\indent \textit{深信} 核武器对人类和对文明的存续造成最大的威胁, \\
\indent \textit{欢迎} 近年来在核裁军和常规裁军方面取得的进展, \\
\indent \textit{确认} 必须保障无核武器国家的独立、领土完整和主权不受使用或威胁使用武力,包括使用或威胁使用核武器的危害, \\
\indent \textit{认为} 在全球实现核裁军之前,国际社会必须制定有效措施和安排,以确保任何方面不使用或威胁使用核武器危害无核武器国家的安全,\\
\indent \textit{铭记} 大会第十届特别会议,即专门讨论裁军问题的第一届特别会议的《最后文件》1 第59 段,其中敦促核武器国家根据情况致力于缔结有效的安排,以保证不对无核武器国家使用或威胁使用核武器,并希望促进执行《最后文件》的有关规定,\\
\indent \textit{回顾} 裁军谈判委员会2提交大会第十二届特别会议、即专门讨论裁军问题的第二届特别会议的特别报告3和裁军谈判会议提交大会第十五届特别会议、即专门讨论裁军问题的第三届特别会议的特别报告4以及裁军谈判会议1992年届会的报告5的有关部分,又回顾载于1980年12月3日第35/46号决议附件内的《宣布1980年代为第二个裁军十年宣言》第12段,其中特别指出,裁军谈判委员会应竭尽全力, 紧急进行谈判,以求就保证不对无核武器国家使用或威胁使用核武器的有效国际安排达成协议, \\
\indent \textit{注意到} 裁军谈判会议及其保证不对无核武器国家使用或威胁使用核武器的有效国际安排特设委员会为了就这项问题达成协议而进行的深入谈判, \\
\indent \textit{又注意到} 2003 年2 月 20 日至 25 日在吉隆坡举行的第十三次不结盟国家国家元首和政府首脑会议的有关决定7 以及伊斯兰会议组织的有关建议, 

\begin{enumerate}
\setlength{\itemsep}{0pt}
\setlength{\parsep}{0pt}
\setlength{\parskip}{0pt}
\item \textit{重申}迫切需要早日就保证不对无核武器国家使用或威胁使用核武器的有效国际安排达成协议; 
\item \textit{注意到}裁军谈判会议中原则上没有人反对缔结一项国际公约以保证不对无核武器国家使用或威胁使用核武器的设想,尽管也有人指出在研拟各方可以接受的共同办法方面存在着困难; 
\item \textit{呼吁}所有国家,特别是核武器国家,就共同办法,特别是可载入具有法律约束力的国际文书的共同方案,积极努力争取及早达成协议; 
\item \textit{建议}进一步加紧努力,寻求这种共同办法或共同方案,并建议进一步探讨各种不同的备选办法,包括特别是在裁军谈判会议上审议的那些办法,以克服各种困难; 
\item \textit{又建议}裁军谈判会议继续积极加紧谈判,以求早日达成协议并缔结关于保证不对无核武器国家使用或威胁使用核武器的有效国际安排,同时考虑到对缔结一项国际公约的广泛支持和为达成这项目标所提出的任何其他提案; 
\item \textit{决定}将题为“缔结关于保证不对无核武器国家使用或威胁使用核武器的有效国际安排”的项目列入大会第六十届会议临时议程。
\end{enumerate}
\end{spacing}

\newpage

\section{修正案(Amendment)}

修正案是用来对正在讨论的决议草案提出修改意见的文件。修正案分为\textbf{友好修正案(Friendly Amendment)}和\textbf{非友好修正案(Unfriendly Amendment)}两种。\textbf{注意,在联合国没有友好修正案和非友好修正案的说法。}修正案必须指明针对的决议草案。

\begin{itemize}

\item 友好修正案(Friendly Amendment)

友好修正案是原决议草案全部起草国均赞成的对决议草案内容的改动。在全部起草国签字同意后,一份修正案便成为友好修正案,并自动被加入到原决议草案中。

\item 非友好修正案(Unfriendly Amendment)

如果原决议草案的起草国未就此修正案达成一致性同意意见,该修正案即成为非友好修正案。非友好修正案需要征集到与会代表的20\%作为附议国签名,才可以向大会主席提交。原决议草案的起草国不能成为非友好修正案的起草国或附议国,原决议草案的附议国则可以签署非友好修正案。修正案表决时需逐条表决(需要2/3多数)。

\end{itemize}

\subsection{修正案的格式}

原决议草案的起草国不可起草修正案但可附议,原决议草案的附议国可起草、附议修正案。修正案格式与草案条款格式基本上一样,在开头的右上角署起草国(即修正案的提出国),如果是友好修正案则附议国为原决议草案的起草国;非友好修正案的附议国则为会代表国家的20\%的署名(除原决议草案起草国)。修正案也是以动词开头,需要分点阐述,第一行要提出对决议的某个条款的修改(增加……、删除……、在……插入……),编号要顺着相应的草案编号进行。

\subsection{修正案范例}
\begin{spacing}{1}
\setlength{\parskip}{0em}
\centerline{\textbf{修正案:1.1}}

\vspace{1em}

\noindent 委员会:大会第一委员会 

\noindent 议题:无核国家的核威胁 

\noindent 起草国:××国 附议国:××国,××国,××国 

\begin{enumerate}
\setlength{\itemsep}{0pt}
\setlength{\parsep}{0pt}
\setlength{\parskip}{0pt}

\item 去掉行动性条款第三项中语句:“特别是核武器国家”; 

\item 加入行动性条款,作为第四项:“倡议所有核武器国家,以全球利益为先,积极寻求可行性途径力求与无核国家达成共识”

\end{enumerate}
\end{spacing}

\section{指令草案(Draft Directive)}

指令草案是针对危机处理的文件。指令草案的表决通过标志着危机的成功解决。指令草案由行动性条款构成。指令草案的行动措施要立即、有效,避免长远规划。代表可以对指令草案提出修正案,其中,非友好修正案也需要表决。

\chapter{主媒体中心}

\section{简介}

对于联合国这一当今世界最重要的多边外交组织而言,媒体力量的重要性不言而喻。国家力量的合作与交锋借助文字、图像、声音与视频,吸引民众关注,改变着全球两百余个国家的政治走向。主媒体中心是由世界各国主要媒体所组成的新闻报道机关。代表们将作为媒体记者参与报道会议进程,用基于事实的报道,一定程度影响会议进程与成果。

\section{代表职责}

为了收集新闻素材并制作稿件,媒体代表在会议期内外往往都要付出巨大的精力,考虑到记者代表工作的性质和要求,主媒体中心并不对一切具体日程做出明确的规定,需要进行集体议程时,相较正式的会议时间也会有所提前或推迟。

媒体代表在会议期间应当完成以下工作:

\begin{enumerate}

\item 在会前研究所代表媒体的利益从属、侧重与写作风格等,并用于新闻制作中;

\item 在会议进行期间,旁听会议、参加新闻发布会、组织代表专访,或在媒体中心进行新闻制作;

\item 摄制会议照片与视频,作为新闻报道的组成部分发表或提交主席团;

\item 在向主席团提交之前,尽可能地确保新闻产品不存在技术、形式和内容上的疏漏;在每日的工作开始前,每位记者应向主席团提交一份文字性质的工作简报。工作简报的内容应当包括对前一阶段新闻成果的总结,并对下一阶段的会议采访与报道计划做出阐述。

\end{enumerate}

在会议期间,主媒体中心的代表可以任意进出会场,并进行旁听、摄影、摄像、采访、报道等活动,但不能干扰会议的正常进行。媒体代表不得在会议进行时邀请国家代表离开会场。

大会期间,主媒体中心还将在特定时间安排特定形式的活动,包括各委员会组织举行的不定期新闻发布会等。除此之外,主席团每天都会将优秀新闻作品收集、整理,以纸质文本或电子出版物形式分发各委员会。

为保证主媒体中心代表新闻写作素材质量和数量,媒体记者代表拥有即时采访权,即选准某个或某几个采访对象,进行面对面的交谈以获得新闻素材。主席团对此不做时间空间上的限制。

\subsection{媒体驻场规则}

会议中为了保证全会场信息畅通及新闻线索的连贯性,每个委员会都将有记者代表进行驻场。会议中,各记者在根据自己中文场或英文场的职责,可以自由串场(限工作语种相同)。

\section{新闻发布会(Press Conference)}

在新闻发布会中,主导者是各个会场的主席团和代表,媒体在其中起到传播信息,发散思维的作用。媒体加快信息的流通,增加与会人员的知情度,及时反馈消息,并且向会场内代表提出问题,拓展其思考方向,从而在一定程度上推进会议的进程,使最终的决策更加周全有效。

新闻发布会的召开时间与召开时长由本次会议的主媒体中心主席团决定。新闻发布会召开前主席团将会告知驻场记者新闻发布会召开的时间地点。记者在每轮发布会前半小时之内需拟好提问大纲或意向问题,并将其交至主新闻中心主席团处。

新闻发布会的形式将由各委员会的主席团决定,但新闻发布会中一定会安排记者提问环节。一旦该会场主席团宣布本次新闻发布会的记者提问环节开始,任何参加新闻发布会的记者都有权提问。由主席团成员点若干记者提问。提问记者的选择,数量,提问环节的时长都由该会场主席团决定。

新闻发布会中,主媒体中心的代表提问形式任意,记者可以提问会场中任意一位代表,甚至是主席团成员,记者不可以提问记者。

关于新闻发布会,有以下规定:

\begin{enumerate}

\item 发布方迟到或缺席,视为主动放弃本次发布会;

\item 发布会开始十五分钟后,MPC 主席团驻场成员有权决定是否结束本次发布会。

\end{enumerate}

\section{出版物规则}

\subsubsection{快讯}

本次会议过程中主媒体中心将建立一个公共新闻平台,平台形式根据会场情况而定。一切简讯,视频播报等其他形式的即时新闻都将在主席团审阅后,在公共新闻平台公布。驻扎英文委员会的记者提交的即时新闻应为英文,并且当\textbf{提前检查并修改语法或者常识性错误}。

\subsubsection{会议的报刊由各报社负责编排}

每家通讯社应设置主编一名,人员职责由主席团指定。在 MPC 主席团设置的截版时间前,每家报社应该将本报社排版完成的报纸版面发送至主席团处。版面要求为A3纸大小,单面,具体内容设计,风格把握,均由报社编辑自行决定。对于报纸内容的来源由各报社编辑选择,主席团建议以各报社的 MPC 代表为主,同时也可接受大会一切参与人员提交的报道。每个记者完成的社论,其语言应当与社论内容所涉及的委员会工作语言一致,并投递到通讯社。主席团欣赏能够提供高数量、高质量,尤其是两者并重的MPC代表。主席团拒绝各个新闻媒体进行新闻共享,主席团不希望看到稿件内容有大面积雷同。为了保证报纸稿件的整体质量,希望各报社编辑在正式排版前把准备使用的稿件发送至主席团处。主席团会进行初步审阅提出修改意见并要求修改后再排版。建议供稿人提前做好必要性修改。主席团不会对稿件实质性内容做出改动。


\chapter{通讯社与报社简介}

\section{报社简介}

\subsection{费加罗报(Le Figaro)}

《费加罗报》隶属于沙克报业集团。该集团在法国和比利时共拥有超过70种法语媒体。其总部位于巴黎。费加罗报是法国的综合性日报,也是法国国内发行量最大的报纸,创刊于1825年。其报名源自法国剧作家博马舍的名剧《费加罗的婚礼》中的主人公费加罗。它的座右铭“倘若批评不自由,则赞美亦无意义(Sans la liberté de blamer, il n'est point d'éloge flatteur)”同样是取自博马舍的这一剧作。政治上主要反映右派乃至右翼保守派的观点,读者以文化水平较高的商界人士和高级职员为主。《费加罗报》在风格理念的变迁以及新闻操作层面的特点上,都更多地强调社会责任与社会引导。这种对大众的“说服”,主要是通过对表达观点的社论进行强化,同时着重新闻元素的选择编排来体现。其一大特点就在于对政府的政策,根据政策领域(修宪问题、防卫政策等)发表旗帜鲜明的社论,带有自民党政权的执政党意识。

\begin{table}[ht]
\setlength{\belowcaptionskip}{5pt}
\caption{《费加罗报》简介}
\centering
\begin{tabular}{|c|c|c|c|}
\hline
中文名称 & 费加罗报      & 主办单位 & 沙克报业集团         \\ \hline
外文名称 & Le Figaro & 创刊时间 & 1826年          \\ \hline
语言   & 法语        & 出版周期 & 日报             \\ \hline
类别   & 综合性       & 国内刊号 & CN11-4637/G0   \\ \hline
主管单位 & 沙克报业集团    & 国际刊号 & ISSN 1671-9689 \\ \hline
\end{tabular}
\end{table}

\subsubsection{主要特点}

与法国另一大报《世界报》标榜言论独立的办报宗旨不同,《费加罗报》在风格理念的变迁以及新闻操作层面的特点上,都更多地强调社会责任与社会引导。这种对大众的“说服”,主要是通过对表达观点的社论进行强化,同时着重新闻元素的选择编排来体现。

\textbf{重视社论,传达报纸的态度与观点}

社论是《费加罗报》的“拳头产品”。19世纪后半期,创办人威尔梅桑将报纸办成巴黎主要的政论报纸之一。此后,《费加罗报》的历任老总都有亲自动笔写社论的传统。著名的有弗朗西斯·马格纳、安托尼·佩里埃等,其评论简短、尖锐,紧扣时事。

副总编夏尔·朗伯斯基尼认为,总编日常工作再忙,也必须坚持撰写社论,给编辑记者作出表率。不同于大仲马、波德莱尔等著名作家的稿件或政治专栏作家雷蒙德·阿伦的时评,这些资深编辑记者的评论直接为报纸立场代言,向读者传达《费加罗报》对新闻事件的态度和看法。

另外,报社专门有一个社论委员会,每篇社论的题目,委员会都要开会商讨。而那些包含著名记者观点的有说服力的文字常被置于显要位置。《费加罗报》通过这种直接灌输来说服读者接受媒介的观点,达到舆论引导的目的。

《费加罗报》社论的主要功能是,对于那些能够帮助理解新闻事件的重要信息,通过社论告诉读者怎样理解。从这个意义上说,社论承担着公共喉舌与社会教化的使命。

《费加罗报》的社论不仅让读者获知新闻事件的相关信息,更重要的是让读者对事件有所感受。①在对社论作者的挑选上也往往倾向于那些法国大众耳熟能详的名字。通过这种针对性极强的战略布置来提升其社论在法国人心中的权威性和影响力。

《费加罗报》多年来通过特殊的版面安排实行一种“预先说服战略”(1999年改版略作调整)。意即,将社论安排在头版头条的左侧,预先决定读者对内页新闻报道的感知角度。同时兼顾逻辑性和文学性,从理性和感性两方面强化读者对文章观点的认同感。文章通常使用复数第一人称,即“我们”,借此涵盖《费加罗报》编辑人员和所有读者,增加贴近性。文末,《费加罗报》常以整个国家的代言人的名义进行呼吁:“我们,法国人民……”“我们将会看到……”等等,并最终落脚于公共利益。

《费加罗报》还不局限于新闻时事的短评,也为讽刺性短文辟出版面,共同引导大众服务。

\textbf{对客观新闻事实进行主观化操作}

媒体通过信息的编辑,形成“聚焦”。《费加罗报》并没有简单地对新闻事实进行罗列,而是遵循一定“等级”进行选择编排,这是一个传者标准与受众需求互相妥协与平衡的过程。需要注意的是,《费加罗报》并没有一味迎合受众,而是在尊重新闻事实的基础上,通过新闻编辑技巧来引导受众进入预定思想空间。

首先,通过对信息的选择与重组,用事实说话。选择感兴趣的新闻素材,将之按一定顺序排列组合,再分别强化和弱化部分信息,从而将主观偏向掩藏在客观的外衣中。对于2008年北京奥运会开幕式的报道,该报在头版头条用醒目标题和图片报道开幕式,在标题中将北京之夜喻为“美丽的仲夏夜之梦”,突出开幕盛况,由此奠定整篇报道的肯定基调。同时,也不忘对开幕前印度、尼泊尔等地反对分子的抗议活动给出几个特写镜头。
《费加罗报》有的报道中偶尔出现少量议论性文字,直接对读者说话,或在文末对纷繁复杂的事件用一句话作深入阐释,或简单明了地使用个别带有修辞成分的短语进行归纳概括。

其次,通过特定写作风格传播倾向。一是运用各种词性和修辞手法。如中国报道标题《令人期待的中国》《中国经济正以龙的步伐向前迈进》等。二是通过描写新闻场景及细节渲染某种气氛来进行暗示。在法国文化年的报道中提到“至于聚餐,法国只能在八达岭停车场举办,那里还停着公共汽车。”寥寥数语的场景描写透露出媒体的不满情绪。通过上述操作方式来强化报道风格,从而对受众进行软性熏陶而非硬性灌输,令传者的价值取向更易被接受。

《费加罗报》之所有会以引导和说服作为办报宗旨,主要有以下几方面原因:

\begin{enumerate}

\item 权利与义务并重的法国自由主义传统。
《费加罗报》虽也标榜客观公正,但蕴含在报道中的主观色彩也较英美媒体更为明显。
\item 强调媒介社会责任功能的社会背景与现实需要。
法国媒体的社会责任功能自20世纪70年代以来有了较大发展。处于国家垄断资本主义阶段的法国,必须通过传播媒介和舆论的社会责任功能这个渠道,干预和指导法国的社会和经济生活。
\item 政党和商业集团利益影响较大
《费加罗报》早期为右翼政党所操纵,成为右派言论阵地。党派观念逐渐淡化后,又一度深陷索克普莱斯集团商业利益的泥沼。

\end{enumerate}

\subsubsection{价值取向}

从历史发展的角度纵观《费加罗报》历任总编的言论行为,可发现该报各时期的编辑理念—创立初期秉持中庸主义;20世纪上半叶崇尚自由言论与社会责任;20世纪下半叶政治立场走向开放,风格趋于现代;21世纪初至今,一方面试图更加独立于利益集团之外,回归新闻本位,另一方面强化新媒体的功能与地位。这种编辑理念的演变,总体而言,是在内容、形式和体制上从保守与僵化平缓过渡到现代与开放。

横向来看,无论《费加罗报》的办报理念与风格在各时期有着怎样的变化,不变的宗旨都是在言论自由的基础上,通过预先说服战略进行社会引导。这是法国特定的媒介生态及社会历史环境决定的,党派及商业利益也对其产生了较大影响。同时,该报以保持其具参考意义的大报形象为目标,这也正是这份百年老报所体现出来的特色。

在坚持特有的办报思想的同时顺应媒体形势变化,正是这份比《纽约时报》更古老的欧洲大报的长寿秘诀。
然而,一味做时代潮流的跟随者而非引领者也令其限于被动局面。在新媒体的挤压吞并和传统媒体的竞争整合逐步升级的新态势下,创新精神与开拓能力的欠缺,某种程度上也成为该报进一步发展的掣肘。

\subsection{卫报(The Guardian)}

《卫报》(The Guardian)是英国的全国性综合内容日报。与《泰晤士报》、《每日电讯报》被合称为英国三大报。由约翰·爱德华·泰勒创办于1821年5月5日。因总部设于曼彻斯特而称为《曼彻斯特卫报》。1959年8月24日改为现名。总部于1964年迁至首都伦敦,不过于曼城和伦敦均设有印刷设施。该报注重报道国际新闻,擅长发表评论和分析性专题文章。一般公众视《卫报》的政治观点为中间偏左。该报主要读者为政界人士、白领和知识分子。

发展至今,《卫报》还一直保留自己的左翼立场(七八十年代非常明显),在英国,人们也把《卫报》戏称为愤青报纸。

\begin{table}[ht]
\centering
\setlength{\belowcaptionskip}{5pt}
\caption{《卫报》简介}
\begin{tabular}{|c|c|c|c|}
\hline
中文名称 & 卫报          & 主管单位   & 卫报和曼彻斯特晚报公司                 \\ \hline
外文名称 & The Guardian & 创刊时间   & 1821年5月5日                   \\ \hline
语言   & 英文          & 出版周期   & 日刊                          \\ \hline
类别   & 报刊          & 官    网 & http://www.theguardian.com/ \\ \hline
\end{tabular}
\end{table}

\subsubsection{办报理念}

报纸的不同政治立场,决定了报纸的价值观,而且影响了他们数字化转型的特点。对《卫报》的讨论,首先需要建立在对这份报纸的传统和历史有所了解,而如果不对这些方面进行研究,就很难了解报纸今天的立场和新闻创新实践。

《卫报》是一张自由民主派的报纸,代表左翼,读者多是知识界和年轻人。在欧洲知识界,《卫报》的影响力超过了任何一张报纸。

\subsection{人民日报(People's Daily)}

《人民日报》(People's Daily)是中国共产党中央委员会机关报。

报纸于1948年6月15日在河北省平山县里庄创刊。时由《晋察冀日报》和晋冀鲁豫《人民日报》合并而成,为华北中央局机关报,同时担负党中央机关报职能。毛泽东同志亲笔为人民日报题写报名。1949年8月1日,中共中央决定人民日报为中国共产党中央委员会机关报,并沿用1948年6月15日的期号。1992年,人民日报被联合国教科文组织评为\textit{世界十大报纸}\footnote{联合国教科文组织在1992年按照影响大小,评定出当前世界十大报纸,依次是:《纽约时报》(美国)、《苏黎世报》(瑞士)、《世界报》(法国)、《泰晤士报》(英国)、《卫报》(英国)、《人民日报》(中国)、《世界报》(德国)、《阿贝塞报》(西班牙)、《华盛顿邮报》(美国)、《真理报》(俄国)。}之一。

人民日报作为党和政府的喉舌,作为中国对外文化交流的重要窗口,作为展现蓬勃发展社会主义新中国的舞台,人民日报积极宣传党和政府的政策主张,记录中国社会的变化,报道中国正在发生的变革。

\begin{table}[ht]
\centering
\setlength{\belowcaptionskip}{5pt}
\caption{《人民日报》简介}
\begin{tabular}{|c|c|c|c|}
\hline
中文名称   & 人民日报           & 国际刊号   & 国内代号:1—96,国外代号:D797  \\ \hline
外文名称   & People's Daily & 邮发代号   & 1-1                         \\ \hline
语言 & 中文             & 总部社址   & 中国北京市朝阳区金台西路2号              \\ \hline
类别 & 中央党报           & 社长 & 李宝善                         \\ \hline
主管单位   & 中国共产党中央委员会     & 总编辑    & 庹震                          \\ \hline
主办单位   & 人民日报社          & ISSN   & 1672-8386                   \\ \hline
编辑单位   & 人民日报社          & 前身 & 《晋察冀日报》                     \\ \hline
创刊时间   & 1948年6月15日     & 报刊版式   & 对开12-16版                    \\ \hline
出版周期   & 每日             & 官网 & 人民网                         \\ \hline
国内刊号   & CN11-0065      & 主要成就   & 世界十大报纸                      \\ \hline
\end{tabular}
\end{table}

\subsubsection{职能解释}

《人民日报》作为党和政府的喉舌,作为中国对外文化交流的重要窗口,作为展现蓬勃发展社会主义新中国的舞台,人民日报积极宣传党和政府的政策主张,记录中国社会的变化,报道中国正在发生的变革。

作为中国共产党的机关报,《人民日报》社论在不同的时期对中国大陆政治都有着决定性的指导作用。社论往往改变了中国大陆的历史。《人民日报》除了为外界提供中国共产党的政策及观点等直接信息外,其社论亦反映了中共中央对事件的处理意见。

《人民日报》通过刊登特约评论员文章,以委婉的手法传递北京官方对国内和国际事务的看法。国内外政情观察家通常会从《人民日报》内文章的一字一句探求中共中央想表达的真正含义以及某些中国所独有的政治讯息。人民日报刊登文章的位置远比文章内容重要。

\subsection{金字塔报(AL Ahram)}

《金字塔报》(AL Ahram)是阿拉伯国家中创办最早的报纸,也是世界最早出现的报纸之一。1875年12月27日在埃及亚历山大创刊,1876年8月5日第一期报纸问世,1900年迁至开罗出版。1968年,《金字塔报》由原来的马兹鲁姆街搬到现址。是一家比较严肃的政治性报纸,每天在夜间10时、凌晨1时30分和3时30分三个时间定版,共出40个版,约10万字的容量。日发行量约110万份。

\begin{table}[ht]
\setlength{\belowcaptionskip}{5pt}
\centering
\caption{《金字塔报》简介}
\begin{tabular}{|c|c|c|c|}
\hline
中文名称 & 金字塔报        & 创刊地点 & 埃及亚历山大              \\ \hline
类别   & 比较严肃的政治性报纸  & 定版时间 & 夜间10时、凌晨1时30分和3时30分 \\ \hline
创刊时间 & 1875年12月27日 & 报刊长度 & 出40个版,约10万字的容量      \\ \hline
出版周期 & 每天          &      &                     \\ \hline
\end{tabular}
\end{table}

\subsection{纽约时报(The New York Times)}

《纽约时报》(The New York Times)有时简称为“时报”(The Times)是一份在美国纽约出版的日报,在全世界发行,有相当高的影响力,美国高级报纸、严肃刊物的代表,长期以来拥有良好的公信力和权威性。由于风格古典严肃,它有时也被戏称为“灰色女士”(The Gray Lady)。它最初的名字是《纽约每日时报》(The New-York Daily Times),创始人是亨利·贾维斯·雷蒙德和乔治·琼斯。

2016年4月18日,《纽约时报》获2016年普利策奖国际报道奖。2016年进入美国发行量前10名。2017年4月10日,《纽约时报》因对俄罗斯总统普京试图扩大俄在海外的影响力的相关报道获得2017年普利策国际报道奖。美国当地时间2018年4月16日,《纽约时报》自由撰稿人Jake Halpern和自由漫画家Michael Sloan荣获2018年度普利策奖社论漫画奖。

\begin{table}[ht]
\centering
\setlength{\belowcaptionskip}{5pt}
\caption{《纽约时报》简介}
\begin{tabular}{|c|c|c|c|}
\hline
中文名称   & 《纽约时报》               & 出版周期   & 日刊                      \\ \hline
外文名称   & 《The New York Times》 & 曾用名    & 《纽约每日时报》                \\ \hline
语    言 & 英语                   & 官    网 & https://www.nytsyn.com/ \\ \hline
类    别 & 时事                   &        &                         \\ \hline
\end{tabular}
\end{table}

\subsubsection{特点}

在新闻报道方面《纽约时报》将自己看做是一份“报纸记录”,这个政策的结果是除纽约当地的新闻外,《纽约时报》很少首先报道一个事件。而假如它真的首先报道一个事件的话,那么这个报道的可靠性是非常高的,因此往往被世界上其它报纸和新闻社直接作为新闻来源。在美国大多数公共图书馆内都提供一份《纽约时报》索引,其内涵是《纽约时报》对时事的报道文章。 《纽约时报》享有可靠的新闻来源的声誉。它的社论一般被认为是比较开通的。不过实际上它的社论是由许多不同的作者撰写的,而他们的观点则从左到右各不相同。

\subsection{产经新闻(Sankei Shimbun)}

《产经新闻》(日语:さんけいしんぶん / Sankei Shimbun)是日本的全国性报纸之一,由富士产经集团旗下的产业经济新闻社(产经新闻社)所发行。其报社宣称每日发行量达到219万,位居全日本第六位。(前五位分别为读卖新闻、朝日新闻、每日新闻、中日新闻和日本经济新闻)。该报的宣传口号是“不避讳,不一窝蜂,讲关键的报纸”,立场较为保守主义,部分人士讽其为“自民党的推销员”。2007年6月27日,产经新闻与微软公司宣布,将从2007年10月起共同开设新闻网站《MSN产经新闻》。

\begin{table}[ht]
\centering
\setlength{\belowcaptionskip}{5pt}
\caption{《产经新闻》简介}
\begin{tabular}{|c|c|c|c|}
\hline
中文名称   & 产经新闻  & 创刊时间   & 1942年11月1日     \\ \hline
语言 & 日语    & 宣传口号   & 不避讳,不一窝蜂,讲关键   \\ \hline
主办单位   & 产经新闻社 & 地位 & 每日发行量位居全日本第六位 \\ \hline
编辑单位   & 产经新闻社 & 政治立场   & 右翼媒体           \\ \hline
\end{tabular}
\end{table}

\section{通讯社简介}

\subsection{美国联合通讯社(The Associated Press)}

美国联合通讯社( 简称美联社,英文:The Associated Press,缩写AP) 是美国最大的通讯社,1846年在芝加哥成立,1893年成为联营公司,1990年将总部迁到纽约。合作伙伴有1700多家报纸,5000多家电视和广播电台;全球有243家新闻分社,在120个国家设有办事处。合众国际社是美国第二大通讯社,1958年由前合众社和国际新闻社合并组成,总部设在佛罗里达州。国外有80多个分社,拥有一个世界范围的图片网。2016年4月18日,美联社获2016年普利策奖公共服务新闻奖。2018年12月,世界品牌实验室编制的《2018世界品牌500强》揭晓,排名第207。

\begin{table}[ht]
\centering
\setlength{\belowcaptionskip}{5pt}
\caption{美国联合通讯社简介}
\begin{tabular}{|c|c|c|c|}
\hline
公司名称 & 美国联合通讯社                   & 成立时间   & 1846年           \\ \hline
外文名称 & The Associated Press (AP) & 公司类型   & 通讯社            \\ \hline
总部地点 & 纽约                        & 官    网 & http://ap.org/ \\ \hline
\end{tabular}
\end{table}


\subsection{路透社(Reuters)}

路透社(Reuters,LSE:RTR,NASDAQ: RTRSY)是世界上最早创办的通讯社之一,也是目前英国最大的通讯社和西方四大通讯社之一。路透社是世界前三大的多媒体新闻通讯社,提供各类新闻和金融数据,在128个国家运行。

路透提供新闻报导给报刊、电视台等各式媒体,并向来以迅速、准确享誉国际。另一方面,路透提供工具和平台,例如股价和外币汇率,让交易员可以分析金融数据和管理交易风险;同时路透的系统让客户可以经由因特网完成买卖,取代电话或是纽约证券交易所的买卖大厅等人工交易方式,它的电子交易服务串连了金融社群。

路透社由德国人保罗·朱利叶斯·路透(Paul Julius Reuter)1850年在德国亚琛创办,次年迁往英国伦敦。1865年,路透把他的私人通讯社扩展成为一家大公司。1916年被改组为路透有限公司。它素以快速的新闻报道被世界各地报刊广为采用而闻名于世。

\begin{table}[ht]
\centering
\setlength{\belowcaptionskip}{5pt}
\caption{路透社简介}
\begin{tabular}{|c|c|c|c|}
\hline
中文名  & 路透社     & 财务长 & 大卫·葛利森              \\ \hline
外文名  & Reuters & 总编辑 & 大卫·史莱辛格             \\ \hline
所属国家 & 英国      & 产业  & 新闻通讯社               \\ \hline
公司类型 & 上市公司    & 产品  & 新闻,金融市场数据与分析        \\ \hline
成立时间 & 1851年   & 营业额 & 24亿9百万英镑(2005年)     \\ \hline
总部   & 英国伦敦    & 员工数 & 16,800人             \\ \hline
董事长  & 尼奥·费兹杰罗 & 子公司 & Instinet            \\ \hline
执行董事 & 汤姆·格罗瑟  & 官网  & http://reuters.com/ \\ \hline
营运长  & 德温卫·韦尼希 &     &                     \\ \hline
\end{tabular}
\end{table}

路透社新闻报道的主要对象是国外,它的国际新闻紧密配合英国政府的外交活动,它对体育新闻也很重视。该社的经济新闻主要是商情报告,为英国和西方大企业服务。
路透社是路透集团的一部分,业务占路透集团的5\%。它素以快速的新闻报道被世界各地报刊广为采用而闻名于世。其股权属于代表伦敦出版的全国性报纸的报业主联合会 (The Newspaper Publishers Association),代表郡级报纸的报联社 (Press Association)、澳大利亚联合新闻社(Australian Associated Press) 和新西兰报联社 (New Zealand Press Association)。这四家股东都有代表参加路透社董事会,它们之间有一项确保路透社在新闻报道中所谓“独立性”和“正直性”的“托拉斯协议”。
路透社的消息大致有特急快讯、急电和普通电讯三种。这三种电讯的时效按顺序递减,篇幅按顺序递增。特急快讯主要针对商业用户,快讯主要适用于政府机关及电子媒介订户,普通电讯则主要服务于其他新闻媒介订户。

\subsection{新华通讯社(Xinhua News Agency)}

新华通讯社,简称新华社,是中国的国家通讯社,法定新闻监管机构,同时也是世界性现代通讯社。新华社是中国共产党早期创建的重要宣传舆论机构,从诞生起就在党中央的直接领导下开展工作,肩负党和人民赋予的神圣使命,发挥喉舌、耳目、智库和信息总汇作用,为党团结带领全国各族人民取得革命、建设和改革的重大胜利作出了重要贡献。新华社在世界各地有一百多个分社,在中国大陆的每个省、直辖市、自治区都设有分社,有的地区还设有支社。新华社是中文媒体的主要新闻来源之一,同时使用英文、法文、西班牙文、俄文、阿拉伯文和葡萄牙文发稿。

新华网是新华社主办的中国重点新闻网站,被称为“中国最有影响力网站”,每天24小时以7种文字、通过多媒体形式不间断地向全球发布新闻信息,全球网站综合排名稳定在190位以内。开通31个地方频道,承办中国政府网、中国平安网、中国文明网、振兴东北网等大型政府网站,形成了中国最大的国家级网站集群。2018年12月,世界品牌实验室编制的《2018世界品牌500强》揭晓,排名第367位。2019年11月8日,新华社获亚通组织卓越通讯社品质奖。

\begin{table}[ht]
\centering
\setlength{\belowcaptionskip}{5pt}
\caption{新华社简介}
\begin{tabular}{|c|c|c|c|}
\hline
中文名  & 新华通讯社                    & 负责范围 & 实时文字新闻 经济信息 图片图表 \\ \hline
外文名  & Xinhua News Agency       & 机构地位 & 中国政府官方的新闻机构      \\ \hline
总部地址 & 北京宣武门西大街57号              & 社长   & 蔡名照              \\ \hline
成立时间 & 1931年11月7日               & 总编辑  & 何平               \\ \hline
官网   & http://www.xinhuanet.com & 隶属关系 & 国务院              \\ \hline
机构级别 & 正部级                      &      &                  \\ \hline
\end{tabular}
\end{table}

文字新闻报道是新华社传统报道形式。它及时、准确、权威地报道党和国家的方针政策及国内外时政、经济、军事、外交、文化等领域的重要新闻。全天24小时滚动发稿,每天播发稿件近600条。新华社文字新闻产品分为6条发稿线路:通稿新闻线路、体育新闻专线、服务新闻专线、财经新闻专线、社会文化新闻专线、专特稿新闻专线。


\subsection{澳大利亚联合新闻社(Australian Associated Press)}

澳联社(Australian Associated Press,简称 AAP)是澳大利亚的一个新闻通讯社,成立于1935年。 澳联社在澳大利亚本土所有省和领地的办事处,聘请超过175名记者。澳联社在莫尔兹比港、雅加达、奥克兰、伦敦、洛杉矶有驻外记者,在欧、美、亚与非洲的其他地方亦有撰稿员;与全球主要国际通讯社均有联盟。澳联社由四大澳大利亚新闻机构拥有,分别是:新闻集团、费尔法克斯媒体、西澳报业(《西澳大利亚州人报》之出版人)和乡郊报业有限公司。新闻集团与费尔法克斯媒体,各自拥有45\%股权,西澳报业占8\%、乡郊报业占2\%。以上4大新闻机构负责出版澳大利亚大部份的报纸。虽然澳联社的主力是突发新闻,但亦有发布软性新闻、故事、专题新闻、意见、填补材料和照片。90年代,澳联社将旗下的电讯部门分拆出来,并成立澳联社电讯公司(AAPT),后来被新西兰电讯所收购。

\subsection{印度报业托拉斯(Press Trust Of India)}

印度报业托拉斯,英文名Press Trust Of India (缩写:PTI),是印度最大的通讯社。其前身是1910年建立的印度联合新闻社。1919年成为路透社的附属机构。1947年印度独立后筹建,1948年正式成立。1949年2月正式发稿。1976年与其他3家通讯社合并为萨尔查尔通讯社。1977年11月,萨马查尔通讯社解散后印度报业托拉斯又恢复工作。

印度报业托拉斯的总社设在孟买,新闻总编室在新德里,为半官方性质,在国内设有132个分社,并在北京、纽约、伦敦、莫斯科等地设有国外分社,在东京、华盛顿、波恩、维也纳、汉城、曼谷等城市设有兼职记者,员工约1500人,国内新闻稿订户450家,与新华社、路透社、法新社等100多家外国新闻机构签有新闻交流协议。印度报业托拉斯为亚通组织成员。

该社是印度报业老板合股企业,凡采用它的消息的印度报刊、电台、电视台,均可购买该社的股票,但股东不分红。

\subsection{泛非通讯社(Pan-African News Agency)}

泛非通讯社,非洲统一组织的通讯社。外文简称PANA。根据1979年7月第十六届非洲统一组织首脑会议决定成立,宗旨是便利成员国之间交换新闻,并为发展和建立非洲各国的通讯社作出贡献。1983年 5月25日开始发布新闻。由14人(国)组成理事会,作为指导性机关。资金由各成员国按一定比例提供 。总部设在塞内加尔首都达喀尔。有工作人员 100 多人,分属28个不同的国籍。1991年参加泛非社新闻交换网的非洲国家通讯社达48个。它在非洲 7 个国家设有分社。消息除一小部分由总社记者采写外,主要由各成员国的通讯社提供。每天用英、法两种文字发稿,约 2.5 万字。各成员国抄收稿件供本国新闻单位采用。

\subsection{德意志新闻社(Deutsche Presse Agentur)}

德意志新闻社(外文名:Deutsche Presse Agentur,简称德新社或DPA)是联邦德国最大的通讯社。1949年9月建于汉堡,是在原美、英、法占领区的3个新闻社的基础上建立起来的。该社是由185家报纸、出版社、电台和电视台等机构联合组成的私人股份有限公司,领导机构是由公司所有成员共同选举的17人组成的监事会。监事会有权决定通讯社的业务领导和主编。德新社共有雇员770多人,其中370多名编辑、记者,在国内外还有数千名特约记者和特约通讯员。在首都波恩设有联邦分社,主要负责联邦一级重大新闻的采编;在慕尼黑、斯图加特、法兰克福、杜塞尔多夫、汉诺威、汉堡以及西柏林设有分社,国内其他地方还有地方分社和采编人员。该社同60多个外国通讯社和新闻单位签订了合作协定,在75个国家有代表,其中一半是常驻记者。该社有订户700多家,40多条专线供稿。1989年5月它与法新社共同创办卫星播送联合公司,播发新闻和图片。1990年又对拉美地区进行卫星直播。德新社播发的新闻包括对国内新闻、对外新闻和图片新闻。对国内新闻包括基础新闻(即对国内播发的较全的新闻)、对州广播的新闻、电话新闻和新闻专稿。对外新闻分别用德、英、西班牙和阿拉伯文对欧洲、拉美、亚洲、非洲、中东和海外6个地区发稿。图片新闻中心设在法兰克福。国际新闻照片主要采用合众国际社的照片。该社自1987年以来,无线电广播业务、图表业务、信息业务办得较成功。

\begin{table}[ht]
\centering
\setlength{\belowcaptionskip}{5pt}
\caption{德新社简介}
\begin{tabular}{|c|c|c|c|}
\hline
中文名    & 德意志新闻社                  & 成立日期 & 1949年9月1日 \\ \hline
外文名    & Deutsche Presse Agentur & 总部   & 汉堡        \\ \hline
简称 & 德新社                     &      &           \\ \hline
\end{tabular}
\end{table}

德新社每天播发约6万字的基础服务,内容涉及国内外政治、经济、科技、社会、文化和体育等各方面的新闻,其中1/3左右为政治新闻。几乎所有德国报纸都接收德新社播发的基础服务,其中1/3的日报靠德新社获得跨地区性的政治新闻,基础服务在日报中的采用率达99\%以上。除报纸外,德国的大多数杂志,以及政府各部门、工会和大型企业也接收德新社的基础服务。

德新社还通过电传、卫星和短波等各种渠道向外国播发基础服务。除了基础服务以外,德新社还把新闻分门别类,为各种专业领域提供特别服务。特别服务的专题有:文化政策、社会政策、环境保护、研究、科学和技术及大众媒介和媒介政策等。接收特别服务的主要是政府部门、工会、高等院校、研究所和各种专业协会。德新社设有图片服务中心,并在德国各主要城市派有摄影记者。德新社还与合众国际社建立了业务联系,接收和处理来自世界各地的照片。图片服务中心拥有一个存放着数百万张各种题材照片的资料库,供用户检索。
德新社播发的新闻主要有:对国内新闻、对外新闻和图片新闻。其中对国内新闻又分为基础新闻、对州广播的新闻、电话新闻和新闻专稿。

德新社的报道和经营由一个17人组成的监事会监督管理,监事会由持股人选举产生。

1974年,德新社与新华社正式签订了交换新闻和合作协议。1996年2月新华社与德新社签订了新的交换新闻和合作协议。

\subsection{韩国联合通讯社(Yonhap News Agency)}

韩国联合通讯社是韩国最大的通讯社。联合通讯社创建于1980年12月19日,总部设在首尔钟路区,全年365天24小时不间断地提供着内容充实的新闻服务。

联合通讯社除了将世界各地的现场消息迅速发回国内,还将国内新闻播发给海外,向全球提供新闻服务。

\begin{table}[ht]
\centering
\setlength{\belowcaptionskip}{5pt}
\caption{韩国联合通讯社简介}
\begin{tabular}{|c|c|c|c|}
\hline
公司名称 & 韩国联合通讯社     & 经营范围   & 新闻服务           \\ \hline
总部地点 & 首尔钟路区       & 地    位 & 韩国的官方通讯社       \\ \hline
成立时间 & 1980年12月19日 & 分    布 & 华盛顿、莫斯科、东京、北京等 \\ \hline
\end{tabular}
\end{table}

\subsubsection{政治影响}

联合通讯社在华盛顿、巴黎、莫斯科、东京、北京等全球16个主要地区派驻了19名特派记者,通过“韩国的眼睛”将世界各地的现场消息迅速发回国内。

除以上5个地区外,联合的海外记者站还有纽约、洛杉矶、香港、曼谷、柏林、布鲁塞尔、开罗、河内、伦敦、上海、日内瓦等。特别是,联合还与美联社、合众社、法新社等西方主要通讯社以及俄罗斯、东欧、中国、日本、中东地区的世界40多家通讯社签订了新闻交流合作协定,发挥着韩国与世界间的桥梁作用。2002年12月还与朝鲜中央通讯社签订了新闻交流合作协定。

联合通讯社还担负着将国内新闻播发给海外的窗口作用。

将政治、经济、社会、文化等各领域的国内重要新闻翻译成英文,向全球提供新闻服务。

联合的英文新闻局目前正向世界40余家外国通讯社、53家韩国驻外使馆和主要国际机构提供英文新闻服务,报道日新月异的韩国社会的各个方面。

联合通讯社在韩国传媒中拥有最大的地方新闻采集网。在全国10个支社派驻了100余名记者,奔波于各地新闻现场,迅速向读者提供内容丰富生动的地方消息。特别是,联合还自封为“提供地方新闻的尖兵”,为迎接地方化时代的到来,正全力以赴提供各种可增进地方居民利益和促进地方社会发展的新闻。

\newpage

\appendix


\chapter{联合国概览}

联合国是一个国际性组织,于1945年成立,现有会员国193个。联合国的宗旨和工作以《联合国宪章》中规定的机构目标和原则为出发点。

由于《宪章》赋予的权利及其独特的国际性质,联合国可就人类在21世纪面临的一系列问题采取行动,具体涉及和平与安全、气候变化、可持续发展、人权、裁军、恐怖主义、人道主义和卫生突发事件、性别平等、施政及粮食生产等。

此外,联合国通过大会、安全理事会、经济及社会理事会和其他机构和委员会,为会员国提供一个论坛来表达他们的观点。并通过促成会员国间对话,主持协商,成为政府间达成协议,携手解决问题的有效机制。

秘书长是联合国的首席行政长官。

2020年是联合国成立七十五周年。

\section{主要机关}

联合国有六个主要机关:大会、安全理事会、经济及社会理事会、托管理事会、国际法院和秘书处,均设立于1945年联合国成立之时。

\subsection{大会}

大会是联合国的主要审议、决策和代表性机关,由联合国全部193个会员国组成,是唯一具有普遍代表性的机关。每年九月,大会的所有会员国齐聚纽约,在联合国大会会议厅召开年度会议,并举行一般性辩论,多国国家元首出席一般性辩论并发表讲话。大会对于重要问题的决定,例如关于和平与安全、接纳新会员国和预算事项的决定,必须由三分之二多数通过;其他问题只须以简单多数通过。大会每年选举一名大会主席,任期一年。

\subsection{安全理事会}

根据《联合国宪章》,安全理事会负有维护国际和平与安全的首要责任。安理会有15个理事国(五个常任理事国和十个非常任理事国),每个理事国有一个投票权。《宪章》规定,所有理事国都有义务履行安理会的决定。安全理事会率先断定对和平的威胁或侵略行为是否存在。安理会促请争端各方以和平手段解决争端,并建议调整办法或解决问题的条件。在某些情况下,安全理事会可实行制裁,甚至授权使用武力,以维护或恢复国际和平与安全。安全理事会设立轮值主席,任期一个月。

\subsection{经济及社会理事会}
经济及社会理事会是就经济、社会和环境问题进行协调、政策审查和政策对话并提出建议,以及落实国际发展目标的主要机关。经社理事会作为联合国全系统开展各项活动的中央机制,其下设立多个涉及经济、社会和环境领域的专门机构、附属监督机构和专家机构。经社理事会共有54个理事国,经大会选举产生,任期三年。经社理事会是联合国对可持续发展问题进行反思、辩论和创新思考的核心平台。

\subsection{托管理事会}
托管理事会于1945年根据《联合国宪章》第十三章设立,对由7个会员国管理的11个托管领土实行国际监督,并确保管理国采取适当措施为托管领土的自治或独立做好准备。截至1994年,所有托管领土都已取得自治或独立。托管理事会于1994年11月1日停止运作。根据1994年5月25日通过的决议,托管理事会对其议事规则作出修正,取消每年举行会议的规定,并同意根据托管理事会或托管理事会主席的决定,或托管理事会多数成员或大会或安全理事会提出的要求,视需要举行全体会议。

\subsection{国际法院}
国际法院是联合国的主要司法机关,位于荷兰海牙的和平宫,是联合国六大主要机关中唯一设在美国纽约之外的机关。国际法院的职责是依照国际法解决各国向其递交的法律争端,并就正式认可的联合国机关和专门机构提交的法律问题提供咨询意见。

\subsection{秘书处}
秘书处由秘书长和在世界各地为联合国工作的数万名国际工作人员组成,负责处理大会和其他主要机关委任的各项日常工作。秘书长是联合国的首席行政长官。联合国从自全球各地招聘国际和当地职员,其工作地点及维和特派团也遍布世界的各个角落。但是在一个暴力的环境下开展维和事业是非常危险的。自联合国成立以来,已有数百名联合国人员在工作中牺牲。

\section{基金、方案和专门机构}

联合国系统包括联合国自身以及被称为方案、基金和专门机构的多个附属组织。这些组织有自己的会员、领导和预算。联合国各方案和基金通过自愿捐助而非分摊会费获得资金。各专门机构是独立的国际组织,并通过自愿捐助和分摊会费获得资金。

\subsection{方案和基金}

\subsubsection{联合国开发计划署}

联合国开发计划署(开发署)在近170个国家和地区开展工作,帮助消除贫困、减少不平等、加强抗灾能力使国家能够维持发展。作为联合国的发展机构,联合国开发计划署在帮助各国实现可持续发展目标过程中起关键作用。

\subsubsection{联合国儿童基金会}

联合国儿童基金会(儿基会)在190个国家和地区开展工作,以拯救儿童的生命,捍卫他们的权利,帮助他们实现自己的潜能。

\subsubsection{世界粮食计划署}

世界粮食计划署(粮食署)致力于根除饥饿和营养不良。作为世界上最大的人道主义机构,粮食署每年向超过75个国家的约8000万人提供粮食援助。

\subsubsection{毒品和犯罪问题办公室}

联合国毒品和犯罪问题办公室旨在帮助会员国打击毒品、犯罪和恐怖主义。

\subsubsection{联合国人口基金}

联合国人口基金(人口基金)作为联合国引领机构,致力于在这个世界实现每一次怀孕都合乎意愿,每一次分娩都安全无恙,每一个青年的潜能都充分实现。

\subsubsection{联合国环境规划署}

联合国环境规划署(环境署)成立于1972年,是联合国系统内负责环境事务的权威机构。环境署激发、提倡、教育和促进全球环境资源的合理利用,并推动全球环境的可持续发展。

\subsubsection{联合国人居署}

联合国人居署致力于促进社会和环境方面可持续性人居发展,以达到所有人都有合适居所的目标。

\subsection{联合国专门机构}

联合国专门机构是通过谈判订立协定与联合国共事的自治组织。有些组织在第一次世界大战前已经存在。有些组织与国际联盟有关。还有些组织几乎与联合国同时创设。其他一些组织则是联合国为满足新需求而设立。

\subsubsection{世界卫生组织}
世界卫生组织(世卫组织)负责全球疫苗接种运动,应对公共卫生紧急情况,防范大流行性流感以及引领致命疾病的根除运动,例如脊髓灰质炎和疟疾。去年,世卫组织消除了越南的禽流感,将两个国家从脊髓灰质炎流行国名单上移除,并向黎巴嫩和达尔富尔提供人道主义援助。

\subsubsection{联合国教育、科学及文化组织}
联合国教育、科学及文化组织(教科文组织)关注包括师资培训、提高全球教育以及保护世界重要历史文化遗产在内的一切事务。今年,教科文组织将28项新增遗址列入《世界遗产名录》,为当代游客和子孙后代保护这些无可取代的珍宝。

\subsubsection{国际劳工组织}
国际劳工组织(劳工组织)通过制定有关结社自由、集体谈判、废除强迫劳动以及机会与待遇平等的国际标准以增进国际劳工权利。

\subsubsection{联合国粮食及农业组织}
联合国粮食及农业组织(粮农组织)领导国际社会为战胜饥饿而努力。它既是发展中国家和发达国家谈判协定的论坛,也为援助发展提供技术知识与信息。

\subsubsection{国际农业发展基金}
国际农业发展基金(农发基金)自1977年成立以来,专注农村减贫工作。基金与发展中国家的农村贫困人口合作以消除贫困、饥饿和营养不良,提高他们的生产力和收入,并提升他们的生活质量。

\subsubsection{国际海事组织}
国际海事组织(海事组织)建立了一个详尽的航运法规框架,处理安全和环境议题、法律事项、技术合作、海事安全和船运效率。

\subsubsection{世界气象组织}
世界气象组织(气象组织)促进气象数据和信息的全球自由交换,进一步提高其在包括航空、航运、安保及农业等一系列事务中的运用。

\subsubsection{世界知识产权组织}
世界知识产权组织(知识产权组织)通过管理23个国际条约保护全世界的知识产权。

\subsubsection{国际民用航空组织}
国际民用航空组织制定国际空运标准,并协助其192个缔约国共同利用世界天空,实现自身社会经济效益。

\subsubsection{国际电信联盟}
国际电信联盟(国际电联)是联合国负责信息和通信技术的专门机构。国际电联致力于连通世界各国人民——无论他们身处何方,处境如何。国际电联通过自身工作,保护并支持每个人的基本通信权利。

\subsubsection{联合国工业发展组织}
联合国工业发展组织(工发组织)是联合国的专门机构,通过促进工业发展来实现减贫、包容性全球化和环境可持续性。

\subsubsection{万国邮政联盟}
万国邮政联盟(万国邮联)是邮政行业参与方之间合作的主要论坛,有助于确保建立一个最新产品与服务的真正全球网络。

\subsubsection{世界旅游组织}
世界旅游组织(世旅组织)是联合国负责促进负责任的、可持续的和人人可享受的旅游业的机构。

\subsubsection{国际货币基金组织}
国际货币基金组织(基金组织)通过为各国提供临时财政援助助其缓解国际收支困难,以及给予技术援助来促进经济增长和创造就业。目前基金组织向全球74个国家发放了280亿贷款。

\subsubsection{世界银行}
世界银行关注减轻贫困和提高全球生活水平,为发展中国家提供低息贷款、无息信贷和赠款,用于教育、卫生、基础设施、通信和其他方面。世界银行在100多个国家开展工作。

\textbf{特殊说明}:国际投资争端解决中心和多边投资保证机构不是根据“联合国宪章”第五十七条和第六十三条设立的专门机构,但是是世界银行集团的一部分。

\subsection{其他联合国实体和机构}

\subsubsection{联合国艾滋病毒/艾滋病联合规划署}

联合国艾滋病毒/艾滋病联合规划署(艾滋病规划署)由难民署、儿基会、粮食计划署、开发署、人口基金、毒品和犯罪问题办公室、劳工组织、教科文组织、妇女署、世卫组织以及世界银行这11个联合国系统机构共同赞助,在终止和应对艾滋病毒/艾滋病传播方面秉承10项目标。

\subsubsection{联合国减少灾害风险办公室}
联合国减少灾害风险办公室(减灾办公室)是联合国系统内减灾协调工作的联络中心。

\subsubsection{联合国难民事务高级专员办事处}
联合国难民事务高级专员办事处(难民署)致力于保护全球难民并帮助他们重返家园或被重新安置。

\subsubsection{联合国裁军研究所}
联合国裁军研究所(裁研所)是联合国系统内由自愿捐款资助的自治机构。研究所立场公正,就裁军和安全问题提出思路,并促进实际行动。裁研所汇集国家、国际组织、民间社会、私营部门和学术界之力,在国际、区域和地方层面建立并实施有益于所有国家和人民的创新解决方案。

\subsubsection{联合国训练研究所}
联合国训练研究所(训研所)成立于1963年,是联合国系统内的一个自治培训机构,旨在通过外交培训提高联合国的效力,并通过公众意识提高、教育和公共政策官员培训,加强国家行动的影响。

\subsubsection{联合国系统职员学院}
联合国系统职员学院是联合国系统内的一个学习机构。学院为联合国系统及其合作伙伴的工作人员设计并实施学习计划,在联合国系统中培养共同的领导和管理文化,从而提高联合国的效率。

\subsubsection{联合国项目事务厅}
联合国项目事务厅是联合国的项目业务部门,它支持世界各地合作伙伴的建设和平、人道主义和发展项目的成功实施。

\subsubsection{联合国近东巴勒斯坦难民救济和工程处}
联合国近东巴勒斯坦难民救济和工程处已在社会福利和人类发展方面服务了四代巴勒斯坦难民。其服务范围涉及教育、医疗保健、救济、社会服务、基础设施和难民营地改善、小额信贷和包括武装冲突期间的紧急援助。工程处只对联合国大会报告其工作情况。

\subsubsection{联合国妇女署}
联合国促进性别平等和增强妇女权能署(妇女署)整合和依托联合国系统内四个原先完全不相干的部门的工作,重点关注两性平等和妇女赋权。

\subsection{相关组织}

\subsubsection{国际原子能机构}
国际原子能机构(原子能机构)是全世界核领域合作的中心,该机构与其成员国和全球多个合作伙伴共同致力于促进核技术的安全、可靠与和平利用。

\subsubsection{世界贸易组织}
世界贸易组织(世贸组织)是各国商榷拟定贸易协定的论坛组织,也为其成员国提供平台,协商应对共同面临的贸易挑战。

\subsubsection{全面禁止核试验条约组织筹备委员会}
全面禁止核试验条约组织筹备委员会(筹委会)致力于推进《全面禁止核试验条约》生效(该条约尚未生效),并建立核查制度,在《条约》生效后予以执行。

\subsubsection{禁止化学武器组织}
禁止化学武器组织(禁化武组织)是1997年生效的《化学武器公约》的执行机构。禁化武组织成员国共同合作,致力于实现无化学武器世界。

\subsubsection{国际移民组织}
国际移民组织(移民组织)旨在确保有序和人道的移徙管理,促进国际社会就移徙议题开展合作,帮助探索切实可行的移徙问题解决方案,为难民、流离失所者等亟需帮助的移徙者提供人道主义援助。

\subsubsection{《联合国气候变化框架公约》}
《联合国气候变化框架公约》秘书处(气候变化秘书处)于1992年各缔约国通过《联合国气候变化框架公约》后成立。随着《京都议定书》和《巴黎协定》相继于1997年和2015年通过,三项协定的缔约方一再重申作为负责支持全球应对气候变化威胁的联合国实体的作用。

\chapter{模拟联合国相关用语}

\section{人员组成}

\subsection*{主席(Chair)}

主席是负责主持辩论、计时、裁决问题及动议,并执行会议流程的主席团成员。

\subsection*{主席团(Dais)}

主席团由主席、学术总监和主席助理组成,负责组织模拟联合国的会议。

\subsection*{学术总监(Director)}

学术总监是负责监督代表角色扮演、文件写作及议题调研的主席团成员。

\subsection*{主席助理(Rapporteur)}

主席团中负责发言名单和点名程序的成员。同时也会帮助主席进行会议进程的记录工作。

\subsection*{秘书处(Secretariat )}

秘书处又称组委会,主要负责模拟联合国会议的学术及会务组织工作,通常由学术总监、代表联系人、会务总监、技术总监等人员组成。

\subsection*{秘书长(Secretary General)}

模拟联合国会议的最高负责人,掌控组委会及主席团的各项工作。

\subsection*{代表(Delegate)}

代表是模拟联合国会议中受委托或指派表达意见的人,可以代表一个联合国成员国,一个观察国,或一个国际组织。

\subsection*{代表团(Delegation)}

代表团是指模联会议中,在不同委员会代表同一个成员国或观察国的全体代表。代表团也指来自同一参与团体的所有代表,例如来自同一高中或大学的全体代表。

\section{会议进程相关}

\subsection*{国家集团(Bloc)}

一组处于同一地理区域或者就某一议题有类似观点的国家。组成国家集团的决定性因素是具有共同的国家利益。

\subsection*{集团领导者(Bloc Leader)}

在一个国家集团中充当着领导者的代表。

\subsection*{成员国(Member State)}

一个批准了联合国宪章、并且加入申请已经被联合国大会和安理会接受的国家。目前联合国一共有192个成员国。唯一一个国际承认的非成员国家是\textit{教廷梵蒂冈}\footnote{拥有梵蒂冈主权的梵蒂冈是联合国观察员国,自1964年4月6日起,教廷即是联合国的常任观察员。}。

\subsection*{起草国(Sponser)}

一份决议草案可以由一个或多个代表来写作,但事实上一份决议草案要得到更多的支持,通常是由多个国家的代表共同起草的,这些国家就被称为起草国,起草国完全同意决议草案里提出的款项。

应特别注意:起草国不能同时成为该决议的附议国,同时在部分模联会场的规则里,不可以附议其他决议草案。

\subsection*{附议国(Signatory)}

希望一份决议草案能够被讨论并且签字支持此行为的国家。附议国并不一定支持此决议,而只是希望能够得到讨论。通常,模联会议对通过一个决议草案有最低起草国和附议国数量的要求。

\subsection*{简单多数(Simple Majority)}

委员会中50\%加1的代表人数。大多数投票所需要的人数。

\subsection*{三分之二多数(Two Thirds Majority)}

议会议事规则中的三分之二多数是指全体出席及有投票权的代表中的三分之二均投赞成票。三分之二多数一般用于决议草案与指令草案的表决。

% \includepdf{2.pdf} 插入背面

\end{document}